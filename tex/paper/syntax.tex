\subsection{Resource annotations}

We assume a partially ordered semiring $(\mathscr{R}, \subres, 0, +, 1, *)$ of
resource annotations.
These will adorn free variables.
$0$ represents non-usage, and $+$ the combination of separate usages.
$(\mathscr{R}, 0, +)$ forms a commutative monoid.
$1$ represents a single or plain usage, and $*$ applies a transformation upon
the type of usage allowed.
$(\mathscr{R}, 1, *)$ forms a monoid that is annihilated by and distributes over
the additive structure.
The order $\subres$ describes a sub-resourcing relation.
We have $\rho \subres \pi$ if and only if annotation $\rho$ is at least as strict as
annotation $\pi$.
$(\mathscr{R}, \subres)$ forms a partial order, and addition and multiplication are
monotonic with respect to $\subres$.

\begin{example}[Linearity]
  We approximate counting usages.
  $0$ and $1$ represent exactly their respective number of usages.
  $\omega$ represents unrestricted usage.
  The effect of the bang ($\oc$) modality of linear logic is achieved by
  multiplication by $\omega$, which makes everything available become
  unrestricted.

  \begin{center}
  \begin{tabular}{>{$}r<{$}|>{$}l<{$}>{$}l<{$}>{$}l<{$}}
    +      & 0      & 1      & \omega \\
    \hline
    0      & 0      & 1      & \omega \\
    1      & 1      & \omega & \omega \\
    \omega & \omega & \omega & \omega \\
  \end{tabular}%
  \hspace{0.5in}%
  \begin{tabular}{>{$}r<{$}|>{$}l<{$}>{$}l<{$}>{$}l<{$}}
    *      & 0      & 1      & \omega \\
    \hline
    0      & 0      & 0      & 0      \\
    1      & 0      & 1      & \omega \\
    \omega & 0      & \omega & \omega \\
  \end{tabular}%
  \hspace{0.5in}%
  \(
    \begin{tikzcd}[arrows=dash,row sep=0.5cm,column sep=0.5cm]
      & \omega & \\
      0 \arrow[ur] && 1 \arrow[ul]
    \end{tikzcd}
  \)
  \end{center}
\end{example}

\begin{example}[Monotonicity]
  Resource annotations track which ``way up'' we can use a free variable.
  A variable can be used covariantly ($\uparrow$), contravariantly
  ($\downarrow$), invariantly ($-$), or unrestrictedly ($?$).

  \begin{center}
  \begin{tabular}{>{$}r<{$}|>{$}l<{$}>{$}l<{$}>{$}l<{$}>{$}l<{$}}
    +          & -          & \uparrow & \downarrow & ? \\
    \hline
    -          & -          & \uparrow & \downarrow & ? \\
    \uparrow   & \uparrow   & \uparrow & ?          & ? \\
    \downarrow & \downarrow & ?        & \downarrow & ? \\
    ?          & ?          & ?        & ?          & ? \\
  \end{tabular}%
  \hspace{0.5in}%
  \begin{tabular}{>{$}r<{$}|>{$}l<{$}>{$}l<{$}>{$}l<{$}>{$}l<{$}}
    *          & - & \uparrow   & \downarrow & ? \\
    \hline
    -          & - & -          & -          & - \\
    \uparrow   & - & \uparrow   & \downarrow & ? \\
    \downarrow & - & \downarrow & \uparrow   & ? \\
    ?          & - & ?          & ?          & ? \\
  \end{tabular}%
  \hspace{0.5in}%
  \(
    \begin{tikzcd}[arrows=dash,row sep=0.5cm,column sep=0.5cm]
      & ? & \\
      \uparrow \arrow[ur] && \downarrow \arrow[ul] \\
      & - \arrow[ul]\arrow[ur] &
    \end{tikzcd}
  \)
  \end{center}

  Notice that $+$ is the join of the sub-resourcing order.
\end{example}

\subsection{Typing \& resourcing}

Figure ??? shows our typing and resourcing system.
(Explain notation)

\begin{mathpar}
  \inferrule*[right=var]
  {(x : B) \in \Gamma
    \\ \subrctx{\underline 0, x^1, \underline 0}{\mathcal R}
  }
  {\ctx{\Gamma}{\mathcal R} \vdash x : B}

  \and

  \inferrule*[right=$\multimap$-I]
  {\ctx{\Gamma}{\mathcal R}, \ctxvar{x}{A}{1} \vdash M : B}
  {\ctx{\Gamma}{\mathcal R} \vdash \lam{x}{M} : \fun{A}{B}}
  \and
  \inferrule*[right=$\multimap$-E]
  {\ctx{\Gamma}{\mathcal P} \vdash M : \fun{A}{B}
    \\ \ctx{\Gamma}{\mathcal Q} \vdash N : A
    \\ \subrctx{\mathcal P + \mathcal Q}{\mathcal R}
  }
  {\ctx{\Gamma}{\mathcal R} \vdash \app{M}{N} : B}

  \inferrule*[right=$\oc_{\rho}$-I]
  {\ctx{\Gamma}{\mathcal P} \vdash M : A
    \\ \subrctx{{\color{black}\rho} * \mathcal P}{\mathcal R}
  }
  {\ctx{\Gamma}{\mathcal R} \vdash \bang{M} : \excl{\rho}{A}}
  \and
  \inferrule*[right=$\oc_{\rho}$-E]
  {\ctx{\Gamma}{\mathcal P} \vdash M : \excl{\rho}{A}
    \\ \ctx{\Gamma}{\mathcal Q}, \ctxvar{x}{A}{\color{black}\rho} \vdash N : B
    \\ \subrctx{\mathcal P + \mathcal Q}{\mathcal R}
  }
  {\ctx{\Gamma}{\mathcal R} \vdash \bm{B}{M}{x}{N} : B}

  \and

  \inferrule*[right=$1$-I]
  {\subrctx{\underline 0}{\mathcal R}}
  {\ctx{\Gamma}{\mathcal R} \vdash \unit : \tensorOne}
  \and
  \inferrule*[right=$1$-E]
  {\ctx{\Gamma}{\mathcal P} \vdash M : \tensorOne
    \\ \ctx{\Gamma}{\mathcal Q} \vdash N : A
    \\ \subrctx{\mathcal P + \mathcal Q}{\mathcal R}
  }
  {\ctx{\Gamma}{\mathcal R} \vdash \del{A}{M}{N} : A}

  \and

  \inferrule*[right=$\otimes$-I]
  {\ctx{\Gamma}{\mathcal P} \vdash M : A
    \\ \ctx{\Gamma}{\mathcal Q} \vdash N : B
    \\ \subrctx{\mathcal P + \mathcal Q}{\mathcal R}
  }
  {\ctx{\Gamma}{\mathcal R} \vdash \ten{M}{N} : \tensor{A}{B}}
  \and
  \inferrule*[right=$\otimes$-E]
  {\ctx{\Gamma}{\mathcal P} \vdash M : \tensor{A}{B}
    \\ \ctx{\Gamma}{\mathcal Q}, \ctxvar{x}{A}{1}, \ctxvar{y}{B}{1} \vdash N : C
    \\ \subrctx{\mathcal P + \mathcal Q}{\mathcal R}
  }
  {\ctx{\Gamma}{\mathcal R} \vdash \prm{C}{M}{x}{y}{N} : C}

  \and

  \inferrule*[right=$\top$-I]
  { }
  {\ctx{\Gamma}{\mathcal R} \vdash \eat : \withTOne}
  \and
  \text{(no $\top$-E)}

  \and

  \inferrule*[right=$\&$-I]
  {\ctx{\Gamma}{\mathcal R} \vdash M : A
    \\ \ctx{\Gamma}{\mathcal R} \vdash N : B
  }
  {\ctx{\Gamma}{\mathcal R} \vdash \wth{M}{N} : \withT{A}{B}}
  \and
  \inferrule*[right=$\&$-E]
  {\ctx{\Gamma}{\mathcal R} \vdash M : \withT{A_0}{A_1}}
  {\ctx{\Gamma}{\mathcal R} \vdash \proj{i}{M} : A_i}

  \text{(no $0$-I)}
  \and
  \inferrule*[right=$0$-E]
  {\ctx{\Gamma}{\mathcal P} \vdash M : \sumTZero
    \\ \subrctx{\mathcal P + \mathcal Q}{\mathcal R}
  }
  {\ctx{\Gamma}{\mathcal R} \vdash \exf{A}{M} : A}

  \inferrule*[right=$\oplus$-I]
  {\ctx{\Gamma}{\mathcal R} \vdash M : A_i}
  {\ctx{\Gamma}{\mathcal R} \vdash \inj{i}{M} : \sumT{A_0}{A_1}}
  \and
  \inferrule*[right=$\oplus$-E]
  {\ctx{\Gamma}{\mathcal P} \vdash M : \sumT{A}{B}
    \\ \ctx{\Gamma}{\mathcal Q}, \ctxvar{x}{A}{1} \vdash N : C
    \\ \ctx{\Gamma}{\mathcal Q}, \ctxvar{y}{B}{1} \vdash O : C
    \\ \subrctx{\mathcal P + \mathcal Q}{\mathcal R}
  }
  {\ctx{\Gamma}{\mathcal R} \vdash \cse{C}{M}{x}{N}{y}{O} : C}
\end{mathpar}

\subsection{Renaming \& substitution}