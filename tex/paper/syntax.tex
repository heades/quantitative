\subsection{Resource annotations}

We assume a partially ordered semiring $(\mathscr{R}, \subres, 0, +, 1, *)$ of
resource annotations.
These will adorn free variables.
$0$ represents non-usage, and $+$ the combination of separate usages.
$(\mathscr{R}, 0, +)$ forms a commutative monoid.
$1$ represents a single or plain usage, and $*$ applies a transformation upon
the type of usage allowed.
$(\mathscr{R}, 1, *)$ forms a monoid that is annihilated by and distributes over
the additive structure.
The order $\subres$ describes a sub-resourcing relation.
We have $\rho \subres \pi$ if and only if annotation $\rho$ is at least as strict as
annotation $\pi$.
$(\mathscr{R}, \subres)$ forms a partial order, and addition and multiplication are
monotonic with respect to $\subres$.

In many cases of interest, $0$ and $+$ are respectively the bottom and join of
the sub-resourcing order, turning it into a join semilattice.
In these cases, with products ($\withT{A}{B}$) coincide with tensor products
($\tensor{A}{B}$), and resource inference is significantly simplified.
(rntz reference?)

\begin{example}[Trivial]
  The one-element partially ordered semiring provides no usage restrictions.
  The calculus becomes equivalent to a simply typed $\lambda$-calculus.
\end{example}

\begin{example}[Linearity]
  We approximate counting usages.
  $0$ and $1$ represent exactly their respective number of usages.
  $\omega$ represents unrestricted usage.
  The effect of the bang ($\oc$) modality of linear logic is achieved by
  multiplication by $\omega$, which makes everything available become
  unrestricted.

  \begin{center}
  \begin{tabular}{>{$}r<{$}|>{$}l<{$}>{$}l<{$}>{$}l<{$}}
    +      & 0      & 1      & \omega \\
    \hline
    0      & 0      & 1      & \omega \\
    1      & 1      & \omega & \omega \\
    \omega & \omega & \omega & \omega \\
  \end{tabular}%
  \hspace{0.5in}%
  \begin{tabular}{>{$}r<{$}|>{$}l<{$}>{$}l<{$}>{$}l<{$}}
    *      & 0      & 1      & \omega \\
    \hline
    0      & 0      & 0      & 0      \\
    1      & 0      & 1      & \omega \\
    \omega & 0      & \omega & \omega \\
  \end{tabular}%
  \hspace{0.5in}%
  \(
    \begin{tikzcd}[arrows=dash,row sep=0.5cm,column sep=0.5cm]
      & \omega & \\
      0 \arrow[ur] && 1 \arrow[ul]
    \end{tikzcd}
  \)
  \end{center}
\end{example}

\begin{example}[Monotonicity]
  Resource annotations track which ``way up'' we can use a free variable.
  A variable can be used covariantly ($\uparrow$), contravariantly
  ($\downarrow$), invariantly ($-$), or unrestrictedly ($?$).

  \begin{center}
  \begin{tabular}{>{$}r<{$}|>{$}l<{$}>{$}l<{$}>{$}l<{$}>{$}l<{$}}
    +          & -          & \uparrow & \downarrow & ? \\
    \hline
    -          & -          & \uparrow & \downarrow & ? \\
    \uparrow   & \uparrow   & \uparrow & ?          & ? \\
    \downarrow & \downarrow & ?        & \downarrow & ? \\
    ?          & ?          & ?        & ?          & ? \\
  \end{tabular}%
  \hspace{0.5in}%
  \begin{tabular}{>{$}r<{$}|>{$}l<{$}>{$}l<{$}>{$}l<{$}>{$}l<{$}}
    *          & - & \uparrow   & \downarrow & ? \\
    \hline
    -          & - & -          & -          & - \\
    \uparrow   & - & \uparrow   & \downarrow & ? \\
    \downarrow & - & \downarrow & \uparrow   & ? \\
    ?          & - & ?          & ?          & ? \\
  \end{tabular}%
  \hspace{0.5in}%
  \(
    \begin{tikzcd}[arrows=dash,row sep=0.5cm,column sep=0.5cm]
      & ? & \\
      \uparrow \arrow[ur] && \downarrow \arrow[ul] \\
      & - \arrow[ul]\arrow[ur] &
    \end{tikzcd}
  \)
  \end{center}

  Notice that $+$ is the join of the sub-resourcing order.
\end{example}

\subsection{Typing \& resourcing}

We take typing contexts $\Gamma, \Delta$ to be vectors of types, and resourcing
contexts $\mathcal P, \mathcal Q, \mathcal R$ to be vectors of resource
annotations.
Informally, we will index both of these by named variables $x, y, z$.
Notation of the form $\Gamma^{\mathcal R}$ represents the combination of a
typing context $\Gamma$ and a resourcing context $\mathcal R$ over the same set
of variables.

Typing judgements are of the form
$\ctxvar{x_1}{A_1}{\rho_1}, \ldots, \ctxvar{x_n}{A_n}{\rho_n} \vdash M : A$.
Each variable in the context carries a resource annotation, but the conclusion
is taken to always have annotation $1$.
Such a restriction is important to keep substitution admissible, as shown in
\cite{quantitative-type-theory}.

Let $\basis x$ be the $x$th basis vector $\vec 0, x^1, \vec 0$ --- that is, the
resource context in which every variable is annotated with $0$ except for $x$,
which is annotated with $1$.
Such a context describes the restriction that semantically all variables except
$x$ are unused, while $x$ is used plainly.
This makes it the principle resourcing of the variable rule.

\[
\inferrule*[right=var]
{(x : B) \in \Gamma
  \\ \subrctx{\basis x}{\mathcal R}
}
{\ctx{\Gamma}{\mathcal R} \vdash x : B}
\]

Allowing $\basis x \subres \mathcal R$, rather than $\basis x = \mathcal R$,
allows less restrictive resourcings.
For example, we are free to use $x$ if its use is unrestricted, and similarly we
are free to discard any variables with unrestricted use.

The following are the rules for $\lambda$-abstractions and application.
$\multimap$-I is standard, except for the restriction that we use the bound
variable plainly.
In $\multimap$-E, the condition $\mathcal P + \mathcal Q \subres \mathcal R$
means that we must be able to share out the resources of $\mathcal R$ into
those used to produce the function ($\mathcal P$) and those used to produce the
argument ($\mathcal Q$).
Sub-resourcing is allowed here \fixme{why?}

\begin{mathpar}
  \inferrule*[right=$\multimap$-I]
  {\ctx{\Gamma}{\mathcal R}, \ctxvar{x}{A}{1} \vdash M : B}
  {\ctx{\Gamma}{\mathcal R} \vdash \lam{x}{M} : \fun{A}{B}}
  \and
  \inferrule*[right=$\multimap$-E]
  {\ctx{\Gamma}{\mathcal P} \vdash M : \fun{A}{B}
    \\ \ctx{\Gamma}{\mathcal Q} \vdash N : A
    \\ \subrctx{\mathcal P + \mathcal Q}{\mathcal R}
  }
  {\ctx{\Gamma}{\mathcal R} \vdash \app{M}{N} : B}
\end{mathpar}

In general, we distinguish between positive products ($\otimes$, read
``tensor'') and negative products ($\&$, read ``with'').

\fixme{Maybe as a lemma after sub-resourcing has been clearly stated
  $\rightarrow$}
As mentioned above, the distinction collapses in the case where addition is the
join of the sub-resourcing order.
This fact can be seen clearly in the introduction rules, where the condition
$\mathcal P + \mathcal Q \subres \mathcal R$ is equivalent to the conjunction of
$\mathcal P \subres \mathcal R$ and $\mathcal Q \subres \mathcal R$.
With these and the admissibility of sub-resourcing, we get the equivalence.

\begin{mathpar}
  \inferrule*[right=$\otimes$-I]
  {\ctx{\Gamma}{\mathcal P} \vdash M : A
    \\ \ctx{\Gamma}{\mathcal Q} \vdash N : B
    \\ \subrctx{\mathcal P + \mathcal Q}{\mathcal R}
  }
  {\ctx{\Gamma}{\mathcal R} \vdash \ten{M}{N} : \tensor{A}{B}}
  \and
  \inferrule*[right=$\otimes$-E]
  {\ctx{\Gamma}{\mathcal P} \vdash M : \tensor{A}{B}
    \\ \ctx{\Gamma}{\mathcal Q}, \ctxvar{x}{A}{1}, \ctxvar{y}{B}{1} \vdash N : C
    \\ \subrctx{\mathcal P + \mathcal Q}{\mathcal R}
  }
  {\ctx{\Gamma}{\mathcal R} \vdash \prm{C}{M}{x}{y}{N} : C}

  \and

  \inferrule*[right=$\&$-I]
  {\ctx{\Gamma}{\mathcal R} \vdash M : A
    \\ \ctx{\Gamma}{\mathcal R} \vdash N : B
  }
  {\ctx{\Gamma}{\mathcal R} \vdash \wth{M}{N} : \withT{A}{B}}
  \and
  \inferrule*[right=$\&$-E$_i$]
  {\ctx{\Gamma}{\mathcal R} \vdash M : \withT{A_0}{A_1}}
  {\ctx{\Gamma}{\mathcal R} \vdash \proj{i}{M} : A_i}
\end{mathpar}

To derive the units for each of the products, we turn what was then binary into
what is now nullary.

\begin{mathpar}
  \inferrule*[right=$1$-I]
  {\subrctx{\vec 0}{\mathcal R}}
  {\ctx{\Gamma}{\mathcal R} \vdash \unit : \tensorOne}
  \and
  \inferrule*[right=$1$-E]
  {\ctx{\Gamma}{\mathcal P} \vdash M : \tensorOne
    \\ \ctx{\Gamma}{\mathcal Q} \vdash N : A
    \\ \subrctx{\mathcal P + \mathcal Q}{\mathcal R}
  }
  {\ctx{\Gamma}{\mathcal R} \vdash \del{A}{M}{N} : A}

  \and

  \inferrule*[right=$\top$-I]
  { }
  {\ctx{\Gamma}{\mathcal R} \vdash \eat : \withTOne}
  \and
  \text{(no $\top$-E)}
\end{mathpar}

Sums are dual to with products.
The behaviour of sums is familiar from intuitionistic type theory, except for
the splitting of resources in the elimination rules between the term being
eliminated and the continuation terms.

\begin{mathpar}
  \text{(no $0$-I)}
  \and
  \inferrule*[right=$0$-E$_i$]
  {\ctx{\Gamma}{\mathcal P} \vdash M : \sumTZero
    \\ \subrctx{\mathcal P + \mathcal Q}{\mathcal R}
  }
  {\ctx{\Gamma}{\mathcal R} \vdash \exf{A}{M} : A}

  \inferrule*[right=$\oplus$-I]
  {\ctx{\Gamma}{\mathcal R} \vdash M : A_i}
  {\ctx{\Gamma}{\mathcal R} \vdash \inj{i}{M} : \sumT{A_0}{A_1}}
  \and
  \inferrule*[right=$\oplus$-E]
  {\ctx{\Gamma}{\mathcal P} \vdash M : \sumT{A}{B}
    \\ \ctx{\Gamma}{\mathcal Q}, \ctxvar{x}{A}{1} \vdash N_0 : C
    \\ \ctx{\Gamma}{\mathcal Q}, \ctxvar{y}{B}{1} \vdash N_1 : C
    \\ \subrctx{\mathcal P + \mathcal Q}{\mathcal R}
  }
  {\ctx{\Gamma}{\mathcal R} \vdash \cse{C}{M}{x}{N_0}{y}{N_1} : C}
\end{mathpar}

Bang

\begin{mathpar}
  \inferrule*[right=$\oc_{\rho}$-I]
  {\ctx{\Gamma}{\mathcal P} \vdash M : A
    \\ \subrctx{{\color{black}\rho} * \mathcal P}{\mathcal R}
  }
  {\ctx{\Gamma}{\mathcal R} \vdash \bang{M} : \excl{\rho}{A}}
  \and
  \inferrule*[right=$\oc_{\rho}$-E]
  {\ctx{\Gamma}{\mathcal P} \vdash M : \excl{\rho}{A}
    \\ \ctx{\Gamma}{\mathcal Q}, \ctxvar{x}{A}{\color{black}\rho} \vdash N : B
    \\ \subrctx{\mathcal P + \mathcal Q}{\mathcal R}
  }
  {\ctx{\Gamma}{\mathcal R} \vdash \bm{B}{M}{x}{N} : B}
\end{mathpar}

\fixme{Constants}

\subsection{Example programs}

\begin{example}[K]
We can show that $\oc_\rho$ satisfies the modal logic axiom K
(that is,
$\fun{\tensor{\excl{\rho}{C}}{\excl{\rho}{D}}}{\excl{\rho}{(\tensor{C}{D})}}$)
for any $\rho$.

\[
  \begin{eqns}
    \lam{bcbd}{& \prm{ }{bcbd}{bc}{bd}{\\
        & \bm{ }{bc}{c}{\\
          & \bm{ }{bd}{d}{\\
            & \bang{\ten{c}{d}}}}}}
  \end{eqns}
\]

In the resourcing derivation, the interesting part is after all of the pattern
matching has been done and we are ready to start building the result.

\[
  \inferrule*[right=$\oc_\rho$-I]{
    \rho * bcbd^0, bc^0, bd^0, c^1, d^1
    \subres bcbd^0, bc^0, bd^0, c^\rho, d^\rho
    \\
    \ctxvar{bcbd}{\tensor{\excl{\rho}{C}}{\excl{\rho}{D}}}{0},
    \ctxvar{bc}{\excl{\rho}{C}}{0}, \ctxvar{bd}{\excl{\rho}{D}}{0},
    \ctxvar{c}{C}{1}, \ctxvar{d}{D}{1}
    \vdash \ten{c}{d} : \tensor{C}{D}
  }
  {
    \ctxvar{bcbd}{\tensor{\excl{\rho}{C}}{\excl{\rho}{D}}}{0},
    \ctxvar{bc}{\excl{\rho}{C}}{0}, \ctxvar{bd}{\excl{\rho}{D}}{0},
    \ctxvar{c}{C}{\rho}, \ctxvar{d}{D}{\rho}
    \vdash \bang{\ten{c}{d}} : \excl{\rho}{(\tensor{C}{D})}
  }
\]

After this move, $c$ and $d$ become amenable to the var rule, which only applies
to super-resources of $1$.

The converse cannot be implemented in general because we get to a stage where we
are essentially trying to prove $\ctxvar{cd}{\tensor{C}{D}}{\rho} \vdash
\wn : \tensor{\excl{\rho}{C}}{\excl{\rho}{D}}$.
At this point, we cannot use the variable $cd$ because we might not have
$1 \subres \rho$, and so the condition of the var rule fails.
We can do $\otimes$-I, but there is no satisfactory way to split $cd^\rho$
between the two halves.
\end{example}

\begin{example}[Comonad]
  We have the following two derivations.

  \begin{mathpar}
    \inferrule*[Right=$\multimap$-I]{
      \inferrule*[Right=$\oc_1$-E]{
        \inferrule*[right=var]{ }
        {\ctxvar{ba}{\excl{1}{A}}{1} \vdash ba : \excl{1}{A}}
        \\
        \inferrule*[Right=var]{ }
        {\ctxvar{ba}{\excl{1}{A}}{0}, \ctxvar{a}{A}{1} \vdash a : A}
      }
      {\ctxvar{ba}{\excl{1}{A}}{1} \vdash \bm{A}{ba}{a}{a} : A}
    }
    {\vdash \lam{ba}{\bm{A}{ba}{a}{a}} : \fun{\excl{1}{A}}{A}}

    \and

    \inferrule*[Right=$\multimap$-I]{
      \inferrule*[Right=$\oc_{\pi * \rho}$-E]{
        \inferrule*[right=var]{ }
        {\ctxvar{ba}{\excl{\pi * \rho}{A}}{1} \vdash ba : \excl{\pi * \rho}}
        \\
        \inferrule*[Right=$\oc_\pi$-I]{
          \inferrule*[Right=$\oc_\rho$-I]{
            \inferrule*[Right=var]{ }
            {\ctxvar{ba}{\excl{\pi * \rho}{A}}{0}, \ctxvar{a}{A}{1} \vdash
              a : A}
          }
          {\ctxvar{ba}{\excl{\pi * \rho}{A}}{0}, \ctxvar{a}{A}{\rho} \vdash
            \bang{a} : \excl{\rho}{A}}
        }
        {\ctxvar{ba}{\excl{\pi * \rho}{A}}{0}, \ctxvar{a}{A}{\pi * \rho} \vdash
          \bang{\bang{a}} : \excl{\pi}{\excl{\rho}{A}}}
      }
      {\ctxvar{ba}{\excl{\pi * \rho}{A}}{1} \vdash
        \bm{A}{ba}{a}{\bang{\bang{a}}} : \excl{\pi}{\excl{\rho}{A}}}
    }
    {\vdash \lam{ba}{\bm{A}{ba}{a}{\bang{\bang{a}}}}
      : \fun{\excl{\pi * \rho}{A}}{\excl{\pi}{\excl{\rho}{A}}}}
  \end{mathpar}

  These are the counit and comultiplication required in showing that $\oc_{(-)}$ is a
  graded comonad in our category of types and functions.
\end{example}

\fixme{exponentials turn additives into multiplicatives}

\subsection{Admissible rules}

\begin{lemma}[Sub-resourcing]
  \[
    \inferrule*[right=subres]{
      \Gamma^{\mathcal Q} \vdash M : A
      \\ \mathcal P \subres \mathcal Q
    }
    {
      \Gamma^{\mathcal P} \vdash M : A
    }
  \]
\end{lemma}
\begin{proof}
  By induction on the resourcing derivation until a rule with a sub-resourcing
  precondition is encountered, in which cases the transitivity of sub-resourcing
  is enough to be done.
\end{proof}

Let $|-|$ denote the set of free variables of a given context.

\begin{definition}[Well resourced substitution]
  A substitution $\sigma$ from $\Gamma^{\mathcal P}$ to
  $\Delta^{\mathcal Q}$ is a tuple with the following data.

  \begin{itemize}
  \item a matrix $\mat \Sigma : \mathscr R^{|\mathcal P| \times |\mathcal Q|}$
  \item a function $M : |\Delta| \to \operatorname{Term} |\Gamma|$
  \item a proof that $\mat \Sigma \mathcal Q \subres \mathcal P$
    \item a proof that $\forall (x:A) \in \Delta.~\Gamma^{\mat \Sigma \basis x}
      \vdash M~x : A$
  \end{itemize}
\end{definition}

\begin{lemma}[Substitution]
  \[
    \inferrule*[right=subst]{
      \Delta^{\mathcal Q} \vdash N : A
      \\ \sigma : \Gamma^{\mathcal P} \Rightarrow \Delta^{\mathcal Q}
    }
    {
      \Gamma^{\mathcal P} \vdash N[\sigma] : A
    }
  \]
\end{lemma}
\renewcommand{\proofname}{Proof sketch}
\begin{proof}
  By induction on the resourcing derivation.
  During the induction, the matrix $\mat \Sigma$ stays constant, and the proof
  that $\mat \Sigma \mathcal Q \subres \mathcal P$ is updated by linearity of
  matrix-vector multiplication.
\end{proof}