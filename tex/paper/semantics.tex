\fixmeBA{We use ``resource'' throughout for the constraints given by the resource typing. Should probably say somewhere that ``resource'' is a stand-in for a more general name that we can't think of.}

Resources will provide a refinement of the semantics of typed programs.
More precisely, we provide a standard semantics of merely well \emph{typed}
terms into $\mathrm{Set}$, which allows us to interpret terms as programs.
Then, our semantics of well \emph{resourced} terms is into indexed binary
relations over elements of the $\mathrm{Set}$ semantics.

In addition to the parametricity of the syntax over the choice of resource
annotations, the semantics is parametrised over a monoidal category of worlds,
which keep track of resources and restrictions we have accumulated.
Also, the choice of how to interpret the base type and the bang modality is
fairly liberal.

\fixmeBA{what is the point of such generality? It is so that we can
  find the commonalities in the applciations, rather than study them
  separately.}

\subsection{Set-theoretic Semantics of typed terms}
\label{sec:set-interp}

We begin with a standard semantics in $\mathrm{Set}$ of typed terms.
We do not require terms to be well resourced at this stage, and consequently
this semantics is entirely oblivious to the concerns of resourcing.
In particular, $\excl{\rho}{}$ has no effect, and the two types of products are
conflated. \fixmeBA{Point out that this semantics will be refined by the resource-aware semantics below}

\begin{displaymath}
  \begin{array}{c@{\hspace{0.5in}}c}
    \begin{eqns}
      \sem{\base} &=& X_\base \\
      \sem{\excl{\rho}{A}} &=& \sem{A} \\
      \sem{\tensorOne} &=& \{*\} \\
      \sem{\withTOne} &=& \{*\} \\
      \sem{\sumTZero} &=& \{\} \\
    \end{eqns}
    &
    \begin{eqns}
      \\
      \sem{\fun{A}{B}} &=& \sem{A} \rightarrow \sem{B} \\
      \sem{\tensor{A}{B}} &=& \sem{A} \times \sem{B} \\
      \sem{\withT{A}{B}} &=& \sem{A} \times \sem{B} \\
      \sem{\sumT{A}{B}} &=& \sem{A} \uplus \sem{B} \\
    \end{eqns}
  \end{array}
\end{displaymath}

Typing contexts $\Gamma$ are interpreted as cartesian products of their points.
Each typing derivation $\Gamma \vdash M : A$ is interpreted as a function
$\sem{M} : \sem{\Gamma} \to \sem{A}$ in the standard way, plus the following
clauses for the bang rules.

\begin{displaymath}
  \begin{eqns}
    \sem{\bang{M}}~\gamma &=& \sem{M}~\gamma \\
    \sem{\bm{C}{M}{x}{N}}~\gamma &=& \sem{N}~(\gamma, \sem{M}~\gamma)
  \end{eqns}
\end{displaymath}

Each constant $c$ must be given a semantics independent of
context.\fixmeBA{spell this out}

\subsection{Resource Semantics}

The refinement resources provide on top of this is a Kripke-indexed logical
relation $\sem{A}^R : \mathcal{W}^{op} \to \operatorname{Rel}\sem{A}$ for each type $A$.
$\mathcal{W}$ is assumed to be a symmetric monoidal category.\fixmeBA{pre-ordered set, and explain what `op' means}.

\fixmeBA{We are interested in worlds because it allows us to track how
  resources affect the meanings of types; we are interested in
  relations because it allows us to control, via the types, what
  observations a program can make about the underlying data.}

\subsubsection{Worlds}

Inspired by modal logic \fixme{ref}, our relational semantics assumes a category
of worlds $\mathcal W$.
The world keeps track of what resources we have allocated to which parts of the
program.
To this end, we need some way of dividing a world into finitely many parts.
The stipulation that $\mathcal W$ is a \emph{symmetric monoidal}
category provides us with this.

This gives a semantic counterpart to tensor products.

\fixmeBA{intuitively: the elements of $\mathcal W$ determine the
  resources that we are talking about, the pre-order determines a
  notion of ``swappability'', and the monoidal product determines what
  it means for two resources to be combined.}

\fixmeBA{Can we get away with only requiring a monoidal pre-ordered
  set here? I think so...}

\fixmeBA{note that in the pre-ordered case, we have more wiggle room
  for what commutative means...}

\begin{example}[Trivial]
  The one element pre-ordered set $\{*\}$, with $* \leq *$ and
  $* \otimes * = *$ gives a ``track nothing'' semantics of
  resources. \fixmeBA{fwd ref the where this is used.}
\end{example}

\newcommand{\append}{\mathop{++}}

\begin{example}[Lists and Permutations]
  Let $K$ be some set of values that will be used as keys for
  sorting. Define $\mathcal{P}_K$ to be the pre-ordered set consisting
  of lists of elements of $K$, with $l_1 \leq l_2$ exactly when $l_1$
  is a permutation of $l_2$. The monoidal product is defined to be
  list concatenation: $l_1 \otimes l_2 = l_1 \append l_2$, with the
  empty list as the monoidal unit.

  \fixmeBA{explain the interpretation of the base type of keys}

  \fixmeBA{explain how this will be used in the $\{0,1,\omega\}$ example.}
\end{example}

\begin{example}[Security Principals]
  Lattice of sets of names of principals. In terms of resources, this
  can be thought of as the collection of principals that are ``happy''
  with this computation. \fixmeBA{still need to fully work out this
    example}
\end{example}

\begin{example}[Distances]
  Positive real numbers, with addition.
\end{example}

\subsubsection{Interpretation of $\multimap$ and $\otimes$}

Given a preorder of semantic resources
$(\mathcal{W}, \leq, \otimes, I)$, we have enough structure to
interpret the function types ($\fun{A}{B}$) and tensor product types
($\tensor{A}{B}$) of \name{}. If we assume we are given semantic
interpretations
$\sem{A}^R : \mathcal{W}^{op} \to \operatorname{Rel}{\sem{A}}$ and
respectively for $B$, then we define the interpretations
$\sem{\fun{A}{B}}^R$, $\sem{\tensor{A}{B}}^R$, and
$\sem{\tensorOne}^R$ as follows:
\begin{displaymath}
  \begin{array}{rllcl}
    \sem{\fun{A}{B}}^R & w & (f,f')
    &\iff& \forall x. \forall a,a'.\sem{A}^Rx~(a,a')
           \Rightarrow \sem{B}^R(w \otimes x)~(f~a,f'~a')
    \\
    \sem{\tensor{A}{B}}^R & w & ((a,b),(a',b'))
    &\iff& \exists x,y.
           w = x \otimes y
           \wedge \sem{A}^Rx~(a,a')
           \wedge \sem{B}^Ry~(b,b')
    \\
    \sem{\tensorOne}^R & w & (*,*) &\iff& w \leq I
  \end{array}
\end{displaymath}

\fixmeBA{this is the ``convolution monoidal product'' or the ``day
  tensor product'' specialised to our situation.}

\fixmeBA{Spell out what this means in terms of (a) permutations, and
  (b) distances. Also point out what happens when the monoidal product
  is actually a join}

\subsubsection{Interpretation of lists}

As mentioned in section XXX, we effectively treat lists as tensored
together. \fixmeBA{spell out what this entails, and relate it to the
  permutations example}

\subsubsection{Interpretation of $\excl{\rho}{}$}

The choice of interpretation of $\excl{\rho}{}$ links the resource
annotations present in \name{}'s typing contexts to the semantics.

\begin{example}[Trivial]
  The identity
\end{example}

\begin{example}[Linear Types]
  $0$ is the identifies everything and forces the resource to be $I$;
  $1$ is the identity; and $\omega$ forces the resource to be
  duplicable.
\end{example}

\begin{example}[Monotonicity Annotations]
  describe this...
\end{example}

\begin{example}[Distances]
  Merp
\end{example}

\subsubsection{Interpretation of the additives: $\with$ and $\oplus$}

Copy the clauses from below.

\begin{displaymath}
  \begin{array}{rllcl}
    \sem{\withTOne}^R & w & (*,*) &\iff& \top \\
    \sem{\withT{A}{B}}^R & w & ((a,b),(a',b')) &\iff&
    \sem{A}^R~w~(a,a') \wedge \sem{B}^R~w~(b,b') \\
    \sem{\sumT{A}{B}}^R & w & (\operatorname{inl} a,\operatorname{inl} a')
                               &\iff& \sem{A}^R~w~(a,a') \\
    \sem{\sumT{A}{B}}^R & w & (\operatorname{inr} b,\operatorname{inr} b')
                               &\iff& \sem{B}^R~w~(b,b') \\
  \end{array}
\end{displaymath}


\subsubsection{Assignment of Relations to Types}

Let $\left| X \right|$ be the proposition that the set $X$ is
inhabited.\fixmeBA{don't need this if we're just using a preorder.}

\fixmeBA{put these in a figure to summarise them, and merge this
  section with the one on the fundamental lemma}

\begin{displaymath}
  \begin{array}{rllcl}
    \sem{\base}^R & w & (c,c') &\iff& R_\base~w~(c,c') \\
    \sem{\fun{A}{B}}^R & w & (f,f')
                               &\iff& \forall x,y.~
                                 \left| \mathcal W(x, y \otimes w) \right|
                                 \Rightarrow \\
    &&&& \forall a,a'.~\sem{A}^R~y~(a,a')
                                 \Rightarrow \sem{B}^R~x~(f~a,f'~a') \\
    \sem{\excl{\rho}{A}}^R & w & (a,a') &\iff& \oc_\rho \sem{A}^R~w~(a,a') \\
    \sem{\tensorOne}^R & w & (*,*) &\iff& \left|\mathcal W(w, I)\right| \\
    \sem{\tensor{A}{B}}^R & w & ((a,b),(a',b'))
                               &\iff& \exists x,y.~
                                 \left| \mathcal W(w, x \otimes y) \right|
                                 \wedge \sem{A}^R~x~(a,a')
                                 \wedge \sem{B}^R~y~(b,b') \\
    \sem{\withTOne}^R & w & (*,*) &\iff& \top \\
    \sem{\withT{A}{B}}^R & w & ((a,b),(a',b')) &\iff&
    \sem{A}^R~w~(a,a') \wedge \sem{B}^R~w~(b,b') \\
    \sem{\sumT{A}{B}}^R & w & (\operatorname{inl} a,\operatorname{inl} a')
                               &\iff& \sem{A}^R~w~(a,a') \\
    \sem{\sumT{A}{B}}^R & w & (\operatorname{inr} b,\operatorname{inr} b')
                               &\iff& \sem{B}^R~w~(b,b') \\
  \end{array}
\end{displaymath}

Type-and-resourcing contexts $\ctx{\Gamma}{R}$ are interpreted as
tensor products of banged points.\fixmeBA{spell this out, and don't use ``points''}

\begin{displaymath}
 \begin{eqns}
   \sem{\ctx{\cdot}{\cdot}}^R~w~(*,*) &=& \mathcal W(w, I) \\
   \sem{\ctx{\Gamma}{R}, \ctxvar{x}{A}{\rho}}^R~w
   ~((\gamma,a), (\gamma',a')) &=& 
 \end{eqns}
 % \sem{\ctxvar{x_1}{A_1}{\rho_1}, \ldots \ctxvar{x_n}{A_n}{\rho_n}} \coloneqq
\end{displaymath}

\subsubsection{Semantic Soundness}

We now give the fundamental semantic soundness result for \name{},
that states that the set-theoretic interpretation of every term
(\autoref{sec:set-interp}) preserves the Kripke relational semantics
of contexts and types described above. The proof of this theorem
proceeds by induction on the typing derivation, and has been
formalised in Agda \autoref{sec:agda-formalisation}.

\begin{theorem}
  If $\ctx{\Gamma}{R} \vdash t : T$, then for all worlds $w$ and
  context interpretations $\gamma, \gamma' \in \sem{\Gamma}$, we have:
  \begin{displaymath}
    \sem{\ctx{\Gamma}{R}}^R w~(\gamma, \gamma') \implies \sem{T}^R w~(\sem{t}\gamma, \sem{t}\gamma')
  \end{displaymath}
\end{theorem}

In the very general setting of this section, it is difficult to see
the significance of this result. In the next section, we will
specialise to instantiations of \name{} with specific resource
semirings and specific semantic settings, which will allow us to
derive concrete semantic results of interest.


% Local Variables:
% TeX-master: "quantitative"
% End:
