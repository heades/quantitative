In this section, we give a brief informal overview of the properties we wish to
obtain of programs for free by restricting usage.
We use this to motivate the design of syntax and semantics detailed in
\autoref{sec:syntax} and \autoref{sec:semantics} respectively.

\subsection{Syntax}

Our fundamental syntactic principle is that we will restrict use of the variable
rule by encoding in contexts \emph{how} its variables can be used.
To each variable in the context, we attach a \emph{usage annotation}.
We may use a given variable only if it can be used in a plain manner, and all
other variables in that context can be unused (discarded).

Unlike types, we should expect the usage annotations of variables to change in a
typing derivation.
For example, suppose we want a linear $\lambda$-calculus, where each
$\lambda$-bound variable has to be used exactly once.
Then in this language, we want to write a curried function of two arguments that
pairs those two arguments together.
Na\"ively, we can write
$\lambda x.\lambda y.~(x, y) : \fun{A}{\fun{B}{\tensor{A}{B}}}$.
When we check this, we have to use $x$ and discard $y$ in the left of the pair,
and use $y$ and discard $x$ in the right.
Doing both of these must constitute using both $x$ and $y$, eventually
discarding neither, implying some notion of accumulation of usages.

To deal with this formally, we can set the usage annotations to be the natural
numbers, with the intention of counting how many times a variable is used.
We can discard a variable annotated $0$, and we can plainly use variables
annotated $1$.
So, to use $x$ in our example, we must be trying to conclude
$\ctxvar{x}{A}{1}, \ctxvar{y}{B}{0} \vdash x : A$, and to use $y$, the
annotations must be the other way round.
Then, forming a pair lets us pointwise add together usage annotations, giving
conclusion
$\ctxvar{x}{A}{1+0}, \ctxvar{y}{B}{0+1} \vdash (x, y) : \tensor{A}{B}$.
In general, we require the set of usage annotations to have an addition
operator, as well as its unit $0$ and designated ``plain use'' annotation $1$.

We also want a way to reify the idea of a variable usable in any particular way
into a value in its own right.
For example, we may want a value that can be used exactly twice, rather than
exactly once.
For this, we introduce the \emph{graded bang} type constructor $\excl{\rho}$,
where $\rho$ is a usage annotation.
This has value constructor $\bang{-}$, and using pattern matching notation
allows us to write $\lambda\bang{x}.~(x, x) : \fun{\excl{2}{A}}{\tensor{A}{A}}$.
Before pattern matching, we have a variable $\ctxvar{b}{\excl{2}{A}}{1}$, and
after pattern matching, it is used up (has annotation $0$) and we get a new
variable $\ctxvar{x}{A}{2}$.

To produce an open term that can be used twice is to produce an open term that
can be used once, but in a context where each variable can be used twice as many
times as it was in producing the term once.
Formally, if
$\ctxvar{x_1}{A_1}{\pi_1}, \ldots, \ctxvar{x_n}{A_n}{\pi_n} \vdash t : B$, then
$\ctxvar{x_1}{A_1}{\rho*\pi_1}, \ldots, \ctxvar{x_n}{A_n}{\rho*\pi_n} \vdash
\bang t : \excl{\rho}{B}$.
This means that we have multiplication on usage annotations, of which $1$ is the
unit.

\subsection{Semantics}

The point of constraining the use of variables is to restrict ourselves to
certain classes of well behaved terms.
For example, we may be interested in the linear terms, or the monotonic terms,
or the terms that are not too sensitive to change.
We say that the terms that use variables in accordance with the rules given by
the syntax and usage annotations are \emph{well provisioned}.
We provide a unified denotational semantics as a tool to show that any well
provisioned term really has the properties we wanted of it.

We start by giving terms a standard $\mathrm{Set}$ semantics, written $\sem
A$, $\sem \Gamma$, and $\sem t : \sem \Gamma \to \sem A$ for types, contexts and
terms, respectively.
When written in a dependently typed programming language, this is an interpreter
that takes a metalanguage value for each variable in the context and produces a
metalanguage value as a result.
If we have a term in context with one variable, we can consider whether $\sem t
: \sem A \to \sem B$ is monotonic.
This would mean that if a bigger $\sem A$ value is given, the resulting $\sem B$
value will also be bigger.
To capture this, we interpret each type $A$ as a binary endorelation over $\sem
A$, written $\sem{A}^R \subseteq \sem A \times \sem A$.
Then, given our one-variable term $\ctxvar{x}{A}{1} \vdash t : B$, its semantics
says that it preserves the relation.
Explicitly, $\sem{t}^R :
\forall a, a'.~\sem{A}^R(a, a') \implies \sem{B}^R(\sem t~a, \sem t~a')$.
% The property we want is the following square, where $\Gamma \vdash t : A$,
% $\gamma, \gamma' \in \sem\Gamma$, $a, a' \in \sem A$.

% \begin{tikzcd}
%   \gamma \arrow[r] \arrow[d, "\sem t" left] & \gamma' \arrow[d, "\sem t"] \\
%   a \arrow[r] & a'
% \end{tikzcd}

For other properties, however, we also need the relation to be preserved
relative to a \emph{world}.
For example, in sensitivity analysis, we want say that a perturbation of
$\varepsilon$ in the environment leads to at most a perturbation of
$\varepsilon$ in the value produced.
The world is this $\varepsilon$.
\fixme{Too early to mention $\otimes$ vs $\&$?}
Additionally, we want the perturbation of an environment to be the sum of
perturbations of its variables, meaning that we need to be able to add worlds
together.
Finally, to work out the perturbation in a variable $\ctxvar{x}{A}{\rho}$, we
must consider the usage annotation $\rho$.
For sensitivity analysis, these annotations are going to be scaling factors, so
the action of an annotation $\rho$ on a world $\varepsilon$ will produce the
world $\rho\varepsilon$.
In general, usage annotations will be interpreted as actions on worlds.
