\documentclass[a4paper]{easychair}

\usepackage{amsmath}
\usepackage{amssymb}
\usepackage{amsthm}
\usepackage{cmll}
\usepackage{mathpartir}
\usepackage{mathrsfs}
\usepackage{xcolor}

\def\newelims{1}
\ifx\newelims\undefined
  \def\newelims{0}
\fi

\newcommand{\bind}[2]{#2}
%\newcommand{\ctx}[2]{{#1}^{\rescomment{#2}}}
\newcommand{\ctx}[2]{{#2}^{#1}}
\newcommand{\ctxvar}[3]{#1 \stackrel{\rescomment{#3}}: #2}
%\newcommand{\rescomment}[1]{{\color{red}#1}}
\newcommand{\rescomment}[1]{{#1}}


\newcommand{\ann}[2]{#1 : #2}
\newcommand{\emb}[1]{\underline{#1}}

\newcommand{\base}[0]{\iota}

\newcommand{\fun}[2]{#1 \multimap #2}
\newcommand{\lam}[2]{\lambda #1.~#2}
\newcommand{\app}[2]{#1~#2}

\newcommand{\excl}[2]{\oc_{#1} #2}
\newcommand{\bang}[1]{\operatorname{bang} #1}
\newcommand{\bm}[4]{\if\newelims0%
\operatorname{bm}_{#1}(#2, \bind{#3}{#4})%
\else%
\mathrm{let}~\bang{#3} = #2~\mathrm{in}~#4 : #1%
\fi}

\newcommand{\tensorOne}[0]{1}
\newcommand{\unit}[0]{{*}_\otimes}
\newcommand{\del}[3]{\if\newelims0%
\operatorname{del}_{#1}(#2, #3)%
\else%
\mathrm{let}~\unit = #2~\mathrm{in}~#3 : #1%
\fi}

\newcommand{\tensor}[2]{#1 \otimes #2}
\newcommand{\ten}[2]{(#1, #2)_{\otimes}}
\newcommand{\prm}[5]{\if\newelims0%
\operatorname{pm}_{#1}(#2, \bind{#3, #4}{#5})%
\else%
\mathrm{let}~\ten{#3}{#4} = #2~\mathrm{in}~#5 : #1%
\fi}

\newcommand{\withTOne}[0]{\top}
\newcommand{\eat}[0]{{*}_{\with}}

\newcommand{\withT}[2]{#1 \with #2}
\newcommand{\wth}[2]{(#1, #2)_{\with}}
\newcommand{\proj}[2]{\operatorname{proj}_{#1} #2}

\newcommand{\sumTZero}[0]{0}
\newcommand{\exf}[2]{\operatorname{ex-falso}_{#1}(#2)}

\newcommand{\sumT}[2]{#1 \oplus #2}
\newcommand{\inj}[2]{\operatorname{inj}_{#1} #2}
\newcommand{\cse}[6]{\if\newelims0%
\operatorname{case}_{#1}(#2, \bind{#3}{#4}, \bind{#5}{#6})%
\else%
\mathrm{case}~#2~\mathrm{of}~\inj{0}{#3} \mapsto #4; \inj{1}{#5} \mapsto #6%
\fi}


\newcommand{\typed}[1]{\mathit{#1t}}
\newcommand{\resourced}[1]{\mathit{#1r}}


\newcommand{\sem}[1]{\llbracket #1 \rrbracket}

%\def\tobar{\mathrel{\mkern3mu  \vcenter{\hbox{$\scriptscriptstyle+$}}%
%                    \mkern-12mu{\to}}}
\newcommand\tobar{\mathrel{\ooalign{\hfil$\mapstochar\mkern5mu$\hfil\cr$\to$\cr}}}


\begin{document}

\title{Linear metatheory via linear algebra}

\author{
  Robert Atkey\inst{1}
  \and
  James Wood\inst{1}\thanks{James Wood is supported by an EPSRC Studentship.}
}

\institute{
  University of Strathclyde,
  Glasgow, United Kingdom \\
  \email{\{robert.atkey,james.wood.100\}@strath.ac.uk}
}

\titlerunning{Linear metatheory via linear algebra}
\authorrunning{Wood and Atkey}

\maketitle

\paragraph{Introduction.}

We introduce a simply typed calculus \name{} that allows the use of variables to
be constrained by usage annotations in the context which binds them.
\name{} is a generalisation of existing core type theories for sensitivity analysis
\cite{reed10distance}, dependency and confidentiality \cite{abadi99core},
linearity \cite{Barber1996}, and modal validity \cite{judgmental}.
It is related to quantitative type theory \cite{quantitative-type-theory}, and
various coeffect calculi \cite{PetricekOM14,BrunelGMZ14,GhicaS14}.

One of our insights is that because our usage annotations form a semiring, we
have just enough structure to talk about vectors and matrices.
We find useful some constructs of linear algebra, culminating in substitution
phrased as application of a linear map.

An earlier version of this work was presented at TyDe 2018
\cite{context-constrained}.
The syntax and semantics are formalised in Agda, with the code at
\url{https://github.com/laMudri/quantitative/}.

\paragraph{Syntax.}

Our syntax is that of a simply typed $\lambda$-calculus modified to let us
reason about how variables are used.
We assume a partially ordered semiring (posemiring) $(\mathscr R, \subres, 0,
+, 1, *)$ of usage annotations, with elements coloured in green for emphasis.
The types are base types ($\base_k$), functions ($\fun{}{}$), tensor products
($\tensorOne$, $\tensor{}{}$), with products ($\withTOne$, $\withT{}{}$), sums
($\sumTZero$, $\sumT{}{}$), and graded bangs ($\excl{\rho}{}$).
A context $\ctx{\Gamma}{R}$ is a combination of a typing context $\Gamma$ and a
usage context $\resctx R$.

\begin{mathpar}
  \rescomment \rho, \rescomment \pi \in \mathscr R \and
  A, B, C \Coloneqq \base_k \mid \fun{A}{B} \mid \tensorOne \mid \tensor{A}{B}
  \mid \withTOne \mid \withT{A}{B} \mid \sumTZero \mid \sumT{A}{B} \mid
  \excl{\rho}{A} \\
  \Gamma, \Delta \Coloneqq \cdot \mid \Gamma, x : A \and
  \resctx P, \resctx Q, \resctx R \Coloneqq
  \cdot \mid \resctx R, x^{\rescomment \rho} \and
  \ctx{\Gamma}{R} \Coloneqq
  \cdot \mid \ctx{\Gamma}{R}, \ctxvar{x}{A}{\rho}
\end{mathpar}

Tensor products are eliminated by pattern matching (each side bound with
annotation $1$), whereas with products are eliminated by projections.
The difference is correspondingly seen in the introduction rules, where the two
halves of a tensor product have separate usage, and the two halves of a with
product have shared usage (illustrated below).

The rule \TirName{$\otimes$-I} is the archetypal use of $+$.
The constraint $\subrctx{\resctx P + \resctx Q}{\resctx R}$ says that $\resctx
R$ must be at least as permissive as the accumulation of usages in $\resctx P$
and $\resctx Q$.
If the addition of the semiring is a join of the order (as in the modality
example below), these two types of product are equivalent.

\begin{mathpar}
  \inferrule*[right=$\otimes$-I]
  {\ctx{\Gamma}{P} \vdash M : A
    \\ \ctx{\Gamma}{Q} \vdash N : B
    \\ \subrctx{\resctx P + \resctx Q}{\resctx R}
  }
  {\ctx{\Gamma}{R} \vdash \ten{M}{N} : \tensor{A}{B}}

  \and

  \inferrule*[right=$\&$-I]
  {\ctx{\Gamma}{R} \vdash M : A
    \\ \ctx{\Gamma}{R} \vdash N : B
  }
  {\ctx{\Gamma}{R} \vdash \wth{M}{N} : \withT{A}{B}}
\end{mathpar}

With the graded bang, we see use of $*$ from the annotation posemiring.
We read $\rescomment \rho * \rescomment \pi$ as applying the action $\rescomment
\rho$ to $\rescomment \pi$.
Introduction can be seen as scaling usage.
Elimination is by pattern matching, where we bind a new variable with whatever
usage annotation the type gave us.

\begin{mathpar}
  \inferrule*[right=$\excl{\rho}{}$-I]
  {\ctx{\Gamma}{P} \vdash M : A
    \\ \subrctx{\rescomment \rho * \resctx P}{\resctx R}
  }
  {\ctx{\Gamma}{R} \vdash \bang{M} : \excl{\rho}{A}}
  \and
  \inferrule*[right=$\excl{\rho}{}$-E]
  {\ctx{\Gamma}{P} \vdash M : \excl{\rho}{A}
    \\ \ctx{\Gamma}{Q}, \ctxvar{x}{A}{\rho} \vdash N : B
    \\ \subrctx{\resctx P + \resctx Q}{\resctx R}
  }
  {\ctx{\Gamma}{R} \vdash \bm{B}{M}{x}{N} : B}
\end{mathpar}

The \TirName{var} rule at $x$ can only be used in a usage context $\resctx R$
when $x$ has a usage annotation as permissive as $1$, and all other variables
have annotation as permissive as $0$.
In other words, $x$ can be used plainly, and all other variables can be
discarded.
This can be succinctly stated as the constraint $\vct e_x \subres \resctx R$,
where $\vct e_x$ is the $x$th basis vector.

\paragraph{Substitution.}

We have two admissible rules leading up to the substitution lemma --- subusaging
(\TirName{subuse}) and weakening (\TirName{weak}) --- stated below.
In the language of linear algebra, weakening is embedding into a space of higher
dimension.

Let $|-|$ denote the length of a context.
Usage contexts are taken to be row vectors.
A \emph{substitution} $\sigma$ from $\ctx{\Gamma}{P}$ to $\ctx{\Delta}{Q}$
comprises a $|\resctx Q| \times |\resctx P|$ matrix $\mat \Sigma$ such that
$\resctx Q \mat \Sigma \subres \resctx P$, and for each $(x:A) \in \Delta$, a
term $M_x$ such that $\Gamma^{\vct e_x \mat \Sigma} \vdash M_x : A$.
Then, the simultaneous substitution lemma is proven via the linearity of
vector-matrix multiplication.

\begin{mathpar}
  \inferrule*[lab=subuse]{
    \ctx{\Gamma}{P} \vdash M : A
    \\ \resctx P \subres \resctx Q
  }
  {
    \ctx{\Gamma}{Q} \vdash M : A
  }

  \and

  \inferrule*[lab=weak]{
    \ctx{\Gamma}{P} \vdash M : A
  }
  {
    \ctx{\Gamma}{P}, \Delta^{\vct 0} \vdash M : A
  }

  \and

  \inferrule*[right=subst]{
    \ctx{\Delta}{Q} \vdash N : A
    \\\\ \sigma : \ctx{\Gamma}{P} \Rightarrow \ctx{\Delta}{Q}
  }
  {
    \ctx{\Gamma}{P} \vdash N[\sigma] : A
  }
\end{mathpar}

The identity substitution, where each variable $x$ is substituted by the term
$x$, is witnessed by the identity matrix.

\paragraph{Specialisations.}
\label{sec:specialisations}

To demonstrate the applicability of \name{}, and give examples of usage
posemirings, we show that it can be specialised to DILL \cite{Barber1996} and
the modal type theory of Pfenning and Davies \cite{judgmental}.

DILL is a linear type theory where contexts are split between unrestricted and
linear variables.
To model linearity, we introduce the $\{\zero, \linear, \unrestricted\}$
posemiring.
Annotation $\zero$ denotes non-use, $\linear$ linear use, and
$\unrestricted$ unrestricted use.
Addition and multiplication are like the corresponding natural number
operations, with $\unrestricted$ acting as an infinite element and $\linear +
\linear = \unrestricted$ in lieu of a $2$ element.
The order is generated by $\zero \subres \unrestricted$ and $\linear \subres
\unrestricted$, with no relation between $\zero$ and $\linear$.
This says that unrestricted variables can be both discarded and used.
We translate a DILL derivation of $\Gamma; \Delta \vdash t : A$ into a \name{}
derivation of $\ctx{\Gamma}{\vct \unrestricted}, \ctx{\Delta}{\vct \linear}
\vdash M_t : A$.
We translate DILL's unannotated $\oc$ into $\excl{\unrestricted}{}$.
In the translation, we make use of \TirName{weak} to ignore $\zero$-use
variables introduced by usage separation.
When translating the other way, we require that $\excl{\zero}{}$ and
$\excl{\linear}{}$ do not occur in the derivation we are translating.
We translate a \name{} derivation of $\ctx{\Gamma}{\vct \unrestricted},
\ctx{\Delta}{\vct \linear}, \ctx{\Theta}{\vct \zero} \vdash M : A$ into a DILL
derivation of $\Gamma; \Delta \vdash t_M : A$.
This makes use of DILL's Environment Weakening lemma to correct cases where a
\name{} subderivation was too precise about usage.

Pfenning and Davies' modal type theory is already stated in the form of usage
annotations.
A variable is annotated either $\true$ or $\valid$.
Furthermore, conclusions are only ever of $\true$ things.
This suggests that $\true$ is the $1$ of the posemiring, and we introduce an
$\unused$ annotation to be the $0$.
The PD variable rule says that both $\true$ and $\valid$ assumptions are
$\true$, so we have $\true \subres \valid$.
Furthermore, all assumptions can be discarded, so $\unused$ is the bottom of the
order.
Addition is the join of this order.
The modality $\square$ is translated to $\excl{\valid}{}$, which tells us that
$\valid * \pi = \valid$ for $\pi \neq \unused$. 
$\unused$ and $\true$ are the annihilator and unit of $*$, respectively.
Having these definitions in place, the translations are similar to those for
DILL.

\paragraph{Semantics.}
We also have a semantics that captures the intensional properties of programs
via families of Kripke indexed relations that refine a simple set-theoretic
semantics.
This allows us to reconstruct the semantic properties of calculi in prior work
for sensitivity analyis \cite{reed10distance}, and dependency and
confidentiality \cite{abadi99core}, as well as a new calculus for monotonicity
analysis.

\bibliographystyle{alpha}
\bibliography{../paper/quantitative}

\end{document}