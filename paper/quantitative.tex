%% For double-blind review submission, w/o CCS and ACM Reference (max submission space)
%\documentclass[sigplan,review,anonymous]{acmart}\settopmatter{printfolios=true,printccs=false,printacmref=false}
%% For double-blind review submission, w/ CCS and ACM Reference
%\documentclass[sigplan,review,anonymous]{acmart}\settopmatter{printfolios=true}
%% For single-blind review submission, w/o CCS and ACM Reference (max submission space)
\documentclass[sigplan,review]{acmart}\settopmatter{printfolios=true,printccs=false,printacmref=false}
%% For single-blind review submission, w/ CCS and ACM Reference
%\documentclass[sigplan,review]{acmart}\settopmatter{printfolios=true}
%% For final camera-ready submission, w/ required CCS and ACM Reference
%\documentclass[sigplan]{acmart}\settopmatter{}


%% Conference information
%% Supplied to authors by publisher for camera-ready submission;
%% use defaults for review submission.
\acmConference[PL'18]{ACM SIGPLAN Conference on Programming Languages}{January 01--03, 2018}{New York, NY, USA}
\acmYear{2018}
\acmISBN{} % \acmISBN{978-x-xxxx-xxxx-x/YY/MM}
\acmDOI{} % \acmDOI{10.1145/nnnnnnn.nnnnnnn}
\startPage{1}

%% Copyright information
%% Supplied to authors (based on authors' rights management selection;
%% see authors.acm.org) by publisher for camera-ready submission;
%% use 'none' for review submission.
\setcopyright{none}
%\setcopyright{acmcopyright}
%\setcopyright{acmlicensed}
%\setcopyright{rightsretained}
%\copyrightyear{2018}           %% If different from \acmYear

%% Bibliography style
\bibliographystyle{ACM-Reference-Format}
%% Citation style
%\citestyle{acmauthoryear}  %% For author/year citations
%\citestyle{acmnumeric}     %% For numeric citations
%\setcitestyle{nosort}      %% With 'acmnumeric', to disable automatic
                            %% sorting of references within a single citation;
                            %% e.g., \cite{Smith99,Carpenter05,Baker12}
                            %% rendered as [14,5,2] rather than [2,5,14].
%\setcitesyle{nocompress}   %% With 'acmnumeric', to disable automatic
                            %% compression of sequential references within a
                            %% single citation;
                            %% e.g., \cite{Baker12,Baker14,Baker16}
                            %% rendered as [2,3,4] rather than [2-4].


%%%%%%%%%%%%%%%%%%%%%%%%%%%%%%%%%%%%%%%%%%%%%%%%%%%%%%%%%%%%%%%%%%%%%%
%% Note: Authors migrating a paper from traditional SIGPLAN
%% proceedings format to PACMPL format must update the
%% '\documentclass' and topmatter commands above; see
%% 'acmart-pacmpl-template.tex'.
%%%%%%%%%%%%%%%%%%%%%%%%%%%%%%%%%%%%%%%%%%%%%%%%%%%%%%%%%%%%%%%%%%%%%%


%% Some recommended packages.
\usepackage{booktabs}   %% For formal tables:
                        %% http://ctan.org/pkg/booktabs
\usepackage{subcaption} %% For complex figures with subfigures/subcaptions
                        %% http://ctan.org/pkg/subcaption

\usepackage{stmaryrd}
\usepackage{mathpartir}
\usepackage{amssymb}
\usepackage{cmll}
\usepackage{xcolor}


\newcommand{\ann}[2]{#1 : #2}
\newcommand{\emb}[1]{\underline{#1}}

\newcommand{\base}[0]{\iota}

\newcommand{\fun}[2]{#1 \multimap #2}
\newcommand{\lam}[2]{\lambda #1. #2}
\newcommand{\app}[2]{#1\ #2}

\newcommand{\excl}[2]{\oc_{#1} #2}
\newcommand{\bang}[1]{\operatorname{bang}\ #1}
\newcommand{\bm}[3]{\operatorname{bm}_{#1}(#2, #3)}

\newcommand{\tensorOne}[0]{1}
\newcommand{\unit}[0]{{*}_\otimes}
\newcommand{\del}[3]{\operatorname{del}_{#1}(#2, #3)}

\newcommand{\tensor}[2]{#1 \otimes #2}
\newcommand{\ten}[2]{(#1, #2)_{\otimes}}
\newcommand{\prm}[3]{\operatorname{pm}_{#1}(#2, #3)}

\newcommand{\withTOne}[0]{\top}
\newcommand{\eat}[0]{{*}_{\with}}

\newcommand{\withT}[2]{#1 \with #2}
\newcommand{\wth}[2]{(#1, #2)_{\with}}
\newcommand{\proj}[2]{\operatorname{proj}_{#1}(#2)}

\newcommand{\sumTZero}[0]{0}
\newcommand{\exf}[2]{\operatorname{ex-falso}_{#1}(#2)}

\newcommand{\sumT}[2]{#1 \oplus #2}
\newcommand{\inj}[2]{\operatorname{inj}_{#1}(#2)}
\newcommand{\cse}[4]{\operatorname{case}_{#1}(#2, #3, #4)}


\newcommand{\bind}[2]{\{#1\}#2}
\newcommand{\ctx}[2]{#1^{\color{red}#2}}
\newcommand{\ctxvar}[3]{#1 \stackrel{{\color{red}#3}}: #2}
\newcommand{\rescomment}[1]{{\color{red}#1}}

\newcommand{\sem}[1]{\llbracket #1 \rrbracket}

\begin{document}

%% Title information
\title%[Short Title]
       {Context Constrained Computation} %% [Short Title] is optional;
                                        %% when present, will be used in
                                        %% header instead of Full Title.
%\titlenote{Working title}             %% \titlenote is optional;
                                        %% can be repeated if necessary;
                                        %% contents suppressed with 'anonymous'
%\subtitle{Subtitle}                     %% \subtitle is optional
%\subtitlenote{with subtitle note}       %% \subtitlenote is optional;
                                        %% can be repeated if necessary;
                                        %% contents suppressed with 'anonymous'


%% Author information
%% Contents and number of authors suppressed with 'anonymous'.
%% Each author should be introduced by \author, followed by
%% \authornote (optional), \orcid (optional), \affiliation, and
%% \email.
%% An author may have multiple affiliations and/or emails; repeat the
%% appropriate command.
%% Many elements are not rendered, but should be provided for metadata
%% extraction tools.

%% Author with single affiliation.
\author{Robert Atkey}
%\authornote{with author1 note}          %% \authornote is optional;
                                        %% can be repeated if necessary
%\orcid{nnnn-nnnn-nnnn-nnnn}             %% \orcid is optional
\affiliation{
  % \position{Position1}
  \department{Computer and Information Sciences}              %% \department is recommended
  \institution{University of Strathclyde}            %% \institution is required
  % \streetaddress{Street1 Address1}
  % \city{City1}
  % \state{State1}
  % \postcode{Post-Code1}
%  \country{UK}                    %% \country is recommended
}
\email{robert.atkey@strath.ac.uk}          %% \email is recommended

%% Author with two affiliations and emails.
\author{James Wood}
% \authornote{with }          %% \authornote is optional;
                                        %% can be repeated if necessary
%\orcid{nnnn-nnnn-nnnn-nnnn}             %% \orcid is optional
\affiliation{
  % \position{Position2a}
  \department{Computer and Information Sciences}              %% \department is recommended
  \institution{University of Strathclyde}            %% \institution is required
  % \streetaddress{Street2a Address2a}
  % \city{City2a}
  % \state{State2a}
  % \postcode{Post-Code2a}
%  \country{UK}                   %% \country is recommended
}
\email{james.wood.100@strath.ac.uk}         %% \email is recommended


%% Abstract
%% Note: \begin{abstract}...\end{abstract} environment must come
%% before \maketitle command
\begin{abstract}
  Coeffects are a way of describing the context in which computation
  takes place. Previous works on coeffect calculi have concentrated on
  the additional capabilities bestowed upon programs operating in
  known contexts. For example, a program that knows that its input are
  streams has the ability to request the $n$th value in that
  stream. In this work, we consider the converse: programs whose
  contexts constrain the computations they an perform.

  FIXME: more detail. And edit it down.
\end{abstract}


%% 2012 ACM Computing Classification System (CSS) concepts
%% Generate at 'http://dl.acm.org/ccs/ccs.cfm'.
\begin{CCSXML}
<ccs2012>
<concept>
<concept_id>10011007.10011006.10011008</concept_id>
<concept_desc>Software and its engineering~General programming languages</concept_desc>
<concept_significance>500</concept_significance>
</concept>
<concept>
<concept_id>10003456.10003457.10003521.10003525</concept_id>
<concept_desc>Social and professional topics~History of programming languages</concept_desc>
<concept_significance>300</concept_significance>
</concept>
</ccs2012>
\end{CCSXML}

\ccsdesc[500]{Software and its engineering~General programming languages}
\ccsdesc[300]{Social and professional topics~History of programming languages}
%% End of generated code


%% Keywords
%% comma separated list
\keywords{keyword1, keyword2, keyword3}  %% \keywords are mandatory in final camera-ready submission


%% \maketitle
%% Note: \maketitle command must come after title commands, author
%% commands, abstract environment, Computing Classification System
%% environment and commands, and keywords command.
\maketitle

\section{Introduction}
\label{sec:introduction}
In normal typed $\lambda$-calculi, variables may be used multiple
times, in multiple contexts, for multiple reasons, as long as the
types agree. The disciplines of linear types \cite{girard87linear} and
coeffects \cite{PetricekOM14,BrunelGMZ14,GhicaS14} refine this by
tracking variable usage. We might track how many times a variable is
used, or if it is used co-, contra-, or invariantly. Such a discipline
yields a general framework of ``context constrained computing'', where
constraints on variables in the context tell us something interesting
about the computation being performed.
% Thus we put the
% type information to work to tell us facts about programs that might
% not otherwise be apparent.

We will present work in progress on capturing the ``intensional''
properties of programs via a family of Kripke indexed relational
semantics that refines a simple set-theoretic semantics of
programs. The value of our approach lies in its generality. We can
accommodate the following examples:
\begin{enumerate}
\item Linear types that capture properties like ``all list
  manipulating programs are permutations''. This example uses the
  Kripke-indexing to track the collection of datums currently being
  manipulated by the program.
\item Monotonicity coeffects that track whether a program uses inputs
  co-, contra-, or in-variantly (or not at all).
\item Sensitivity typing, tracking the sensitivity of programs in
  terms of input changes. This forms the core of systems for
  differential privacy \cite{reed10distance}.
\item Information flow typing, in the style of the Dependency Core
  Calculus \cite{abadi99core}.
\end{enumerate}
Through discusssion at the workshop, we hope to discover more
applications of our framework. In future work, we plan to extend our
framework with type dependency, and to explore the space of inductive
data types and elimination principles possible in the presence of
usage information.

The syntax and semantics we present here have been formalised in Agda:
\url{https://github.com/laMudri/quantitative/}. Formalisation of the
examples is in progress.

% Local Variables:
% TeX-master: "quantitative"
% End:


\section{Syntax}
\label{sec:syntax}
\subsection{Typing}
\begin{mathpar}
  \inferrule*[right=var]
  {(x : T) \in \Gamma}
  {\Gamma \vdash x \in T}

  \and

  \inferrule*[right=cod]
  {\Gamma \vdash e \in S \\ S <: T}
  {\Gamma \vdash T \ni \emb{e}}
  \and
  \inferrule*[right=doc]
  {\Gamma \vdash S \ni s}
  {\Gamma \vdash \ann{s}{S} \in S}

  \and

  \inferrule*[right=$\multimap$-I]
  {\Gamma, x : S \vdash T \ni t}
  {\Gamma \vdash \fun{S}{T} \ni \lam{x}{t}}
  \and
  \inferrule*[right=$\multimap$-E]
  {\Gamma \vdash f \in \fun{S}{T} \\ \Gamma \vdash S \ni s}
  {\Gamma \vdash \app{f}{s} \in T}

  \inferrule*[right=$\oc_\rho$-I]
  {\Gamma \vdash S \ni s}
  {\Gamma \vdash \excl{\rho}{S} \ni \bang{s}}
  \and
  \inferrule*[right=$\oc_\rho$-E]
  {\Gamma \vdash e \in \excl{\rho}{S} \\ \Gamma, x : S \vdash T \ni t}
  {\Gamma \vdash T \ni \bm{T}{e}{x}{t}}

  \and

  \inferrule*[right=$1$-I]
  { }
  {\Gamma \vdash \tensorOne \ni \unit}
  \and
  \inferrule*[right=$1$-E]
  {\Gamma \vdash e \in \tensorOne \\ \Gamma \vdash S \ni s}
  {\Gamma \vdash \del{S}{e}{s} \in S}

  \and

  \inferrule*[right=$\otimes$-I]
  {\Gamma \vdash S \ni s \\ \Gamma \vdash T \ni t}
  {\Gamma \vdash \tensor{S}{T} \ni \ten{s}{t}}
  \and
  \inferrule*[right=$\otimes$-E]
  {\Gamma \vdash e \in \tensor{S}{T} \\ \Gamma, x : S, y : T \vdash U \ni u}
  {\Gamma \vdash \prm{U}{e}{x}{y}{u} \in U}

  \and

  \inferrule*[right=$\top$-I]
  { }
  {\Gamma \vdash \withTOne \ni \eat}
  \and
  \text{(no $\top$-E)}

  \and

  \inferrule*[right=$\&$-I]
  {\Gamma \vdash S \ni s \\ \Gamma \vdash T \ni t}
  {\Gamma \vdash \withT{S}{T} \ni \wth{s}{t}}
  \and
  \inferrule*[right=$\&$-E]
  {\Gamma \vdash e \in \withT{S_0}{S_1}}
  {\Gamma \vdash \proj{i}{e} \in S_i}

  \text{(no $0$-I)}
  \and
  \inferrule*[right=$0$-E]
  {\Gamma \vdash e \in \sumTZero}
  {\Gamma \vdash \exf{S}{e} \in S}

  \inferrule*[right=$\oplus$-I]
  {\Gamma \vdash S_i \ni s}
  {\Gamma \vdash \sumT{S_0}{S_1} \ni \inj{i}{s}}
  \and
  \inferrule*[right=$\oplus$-E]
  {\Gamma \vdash e \in \sumT{S}{T}
    \\ \Gamma, x : S \vdash U \ni s \\ \Gamma, y : T \vdash U \ni t}
  {\Gamma \vdash \cse{U}{e}{x}{s}{y}{t} \in U}
\end{mathpar}

\subsection{Resourcing}
\begin{mathpar}
  \inferrule*[right=var]
  {\Delta \leq \underline 0, x^\rho, \underline 0}
  {\Delta \vdash x}

  \and

  \inferrule*[right=cod]
  {\Delta \vdash \typed e}
  {\Delta \vdash \emb{\typed e}}
  \and
  \inferrule*[right=doc]
  {\Delta \vdash \typed s}
  {\Delta \vdash \ann{\typed s}{S}}

  \and

  \inferrule*[right=$\multimap$-I]
  {\Delta, x^1 \vdash \typed t}
  {\Delta \vdash \lam{x}{\typed t}}
  \and
  \inferrule*[right=$\multimap$-E]
  {\Delta_f \vdash \typed f \\ \Delta_s \vdash \typed s
    \\ \Delta \leq \Delta_f + \Delta_s}
  {\Delta \vdash \app{\typed f}{\typed s}}

  % NOTE: multiplication on the right
  % This should change to multiplication on the left,
  % at least here, and probably in the formalisation.
  \inferrule*[right=$\oc_\rho$-I]
  {\Delta_s \vdash \typed s \\ \Delta \leq \Delta_s * \rho}
  {\Delta \vdash \bang{\typed s}}
  \and
  \inferrule*[right=$\oc_\rho$-E]
  {\Delta_e \vdash \typed e \\ \Delta_t, x^\rho \vdash \typed t
    \\ \Delta \leq \Delta_e + \Delta_t}
  {\Delta \vdash \bm{T}{\typed e}{x}{\typed t}}

  \and

  \inferrule*[right=$1$-I]
  {\Delta \leq \underline 0}
  {\Delta \vdash \unit}
  \and
  \inferrule*[right=$1$-E]
  {\Delta_e \vdash \typed e \\ \Delta_s \vdash \typed s
    \\ \Delta \leq \Delta_e + \Delta_s}
  {\Delta \vdash \del{S}{\typed e}{\typed s}}

  \and

  \inferrule*[right=$\otimes$-I]
  {\Delta_s \vdash \typed s \\ \Delta_t \vdash \typed t
    \\ \Delta \leq \Delta_s + \Delta_t}
  {\Delta \vdash \ten{\typed s}{\typed t}}
  \and
  \inferrule*[right=$\otimes$-E]
  {\Delta_e \vdash \typed e \\ \Delta_s, x^1, y^1 \vdash \typed u
    \\ \Delta \leq \Delta_e + \Delta_s}
  {\Delta \vdash \prm{U}{\typed e}{x}{y}{\typed u}}

  \and

  \inferrule*[right=$\top$-I]
  { }
  {\Delta \vdash \eat}
  \and
  \text{(no $\top$-E)}

  \and

  \inferrule*[right=$\&$-I]
  {\Delta \vdash \typed s \\ \Delta \vdash \typed t}
  {\Delta \vdash \wth{\typed s}{\typed t}}
  \and
  \inferrule*[right=$\&$-E]
  {\Delta \vdash \typed e}
  {\Delta \vdash \proj{i}{\typed e}}

  \text{(no $0$-I)}
  \and
  \inferrule*[right=$0$-E]
  {\Delta_e \vdash \typed e \\ \Delta \leq \Delta_e + \Delta_{st}}
  {\Delta \vdash \exf{S}{\typed e}}

  \inferrule*[right=$\oplus$-I]
  {\Delta \vdash \typed s}
  {\Delta \vdash \inj{i}{\typed s}}
  \and
  \inferrule*[right=$\oplus$-E]
  {\Delta_e \vdash \typed e
    \\ \Delta_{st}, x^1 \vdash \typed s \\ \Delta_{st}, y^1 \vdash \typed t
    \\ \Delta \leq \Delta_e + \Delta_{st}}
  {\Delta \vdash \cse{U}{\typed e}{x}{\typed s}{y}{\typed t}}
\end{mathpar}


\section{Semantics}
\label{sec:semantics}
We start by giving a standard semantics of types and well typed (but not necessarily well resourced) terms into sets.

\begin{displaymath}
  \begin{array}{ll}
    \llbracket \base \rrbracket = A_\base \\
    \llbracket \fun{S}{T} \rrbracket = \llbracket S \rrbracket \rightarrow \llbracket T \rrbracket &
    \llbracket \excl{\rho}{S} \rrbracket = \llbracket S \rrbracket \\
    \llbracket \tensorOne \rrbracket = \llbracket \withTOne \rrbracket = \{*\} &
    \llbracket \tensor{S}{T} \rrbracket = \llbracket \withT{S}{T} \rrbracket = \llbracket S \rrbracket \times \llbracket T \rrbracket \\
    \llbracket \sumTZero \rrbracket = \{\} &
    \llbracket \sumT{S}{T} \rrbracket = \llbracket S \rrbracket \uplus \llbracket T \rrbracket \\
  \end{array}
\end{displaymath}

The semantics of contexts and terms are as usual.

Using this, we give a semantics of types $T$ into functors from $\mathcal{W}^{op}$ to $\mathrm{Rel}\ \llbracket T \rrbracket$, where $\mathcal{W}$ is a symmetric promonoidal category and $\mathrm{Rel}\ A$ is the category of binary relations on $A$.

\begin{displaymath}
  \begin{array}{lll}
    \llbracket \base \rrbracket^R & = & R_\base b\\
    \llbracket \fun{S}{T} \rrbracket^R w (f, f') & = & \forall x,y. P(y,w)x \implies \phantom{X} \\
    \quad \forall s,s'. \llbracket S \rrbracket^R y (s, s') \implies \llbracket T \rrbracket^R x (f\ s, f'\ s') \span \span \\
    \llbracket \tensorOne \rrbracket^R w (*, *) & = & J w \\
    \llbracket \tensor{S}{T} \rrbracket^R w ((s, t), (s', t')) & = & \\
    \quad \exists x,y. P(x,y)w \wedge \llbracket S \rrbracket^R x (s, s') \wedge \llbracket T \rrbracket^R y (t, t') \span \span \\
    \llbracket \withTOne \rrbracket^R w (*, *) & = & \top \\
    \llbracket \withT{S}{T} \rrbracket^R w & = & \llbracket S \rrbracket^R w \times_R \llbracket T \rrbracket^R w \\
    \llbracket \sumTZero \rrbracket^R w (s, s') & \mathrm{impossible} \span \\
    \llbracket \sumT{S}{T} \rrbracket^R w & = & \llbracket S \rrbracket^R w \uplus_R \llbracket T \rrbracket^R w \\
    \llbracket \excl{\rho}{S} \rrbracket^R & = & \oc_\rho \llbracket S \rrbracket^R \\
  \end{array}
\end{displaymath}

$R \times_R S$ and $R \uplus_R S$ are defined pointwise.
We assume a family of natural transformations $\oc$ satisfying the following laws.

\begin{displaymath}
  \begin{array}{l}
    \rho \leq \pi \implies (\oc_\pi R\ w \implies \oc_\rho R\ w) \\
    \oc_0 R\ w \implies J w \\
    \oc_{\rho+\pi} R\ w\ (a, b) \implies \exists x,y. P(x,y)w \wedge \oc_\rho R\ x\ a \wedge \oc_\pi R\ y\ b \\
    \oc_1 R\ w \iff R\ w \\
    \oc_{\rho \cdot \pi} R\ w \iff \oc_\rho(\oc_\pi R)\ w
  \end{array}
\end{displaymath}

The semantics of a context $\ctxvar{x_1}{S_1}{\rho_1}, \ldots, \ctxvar{x_n}{S_n}{\rho_n}$ is given by $\tensor{\llbracket \excl{\rho_1}{S_1}}{\tensor{\ldots}{\excl{\rho_n}{S_n}}} \rrbracket^R$.

This indexed relational semantics gives us a family of logical relations.
The fundamental lemma is as follows.

\begin{displaymath}
  \ctx{\Gamma}{\Delta} \vdash t : S \implies \llbracket \ctx{\Gamma}{\Delta} \rrbracket^R w\ (\gamma, \gamma') \implies \llbracket t \rrbracket^R w\ (s, s')
\end{displaymath}


% Local Variables:
% TeX-master: "quantitative"
% End:


\section{Example Instantiations}
\label{sec:examples}
The ingredients of our fundamental lemma are perhaps well known
(relational interpretations, Kripke-indexing), but the value of our
framework lies in the generality of being able to choose $\mathcal{W}$
and its promonoidal structure, and the interpretion the $\oc_\rho$
modality as a relation transformer. Examples include:

\vspace{-0.6em}
 
\paragraph{Permutation Types} With the $\{0,1,\omega\}$ semiring, we
take the category $\mathcal{W}$ to consist of lists of some type of
keys, and permutations between them. The relation transformer is
defined as: $\oc_0 R~l = \top$, where $\top$ is the total relation,
$\oc_1 R~l = R~l$ and $R_\omega~R~l = (l = []) \land R~l$. With
suitable types of keys and lists of keys, the fundamental lemma states
that all programs are permutations. This result has already been
formalised in a one-off type system at
\url{https://github.com/bobatkey/sorting-types}.

\vspace{-0.6em}
 
\paragraph{Monotonicity Types} With $R$ the partially ordered semiring
with carrier $\{0,\uparrow,\downarrow,\equiv\}$ ordered
${\equiv} \leq {\uparrow},{\downarrow}$ and ${\uparrow}, {\downarrow} \leq 0$,
we take $\mathcal{W}$ to be the one-object, one-arrow category, and
define the relation transformer $\oc$ to be:
\begin{mathpar}
  \oc_0~R = \top

  \oc_\uparrow~R = R

  \oc_\downarrow~R = R^{op}

  \oc_\equiv~R = R \cap R^{op}
\end{mathpar}
If we let our base type be natural numbers with the relational
interpretation $R_{\mathrm{nat}}(n,n') \Leftrightarrow n \leq n'$,
then the fundamental lemma states that a program of type
$\ctxvar{x}{\mathrm{nat}}{\uparrow} \vdash t : \mathrm{nat}$ is
covariant (and similarly for contravariant and invariant).

\vspace{-0.6em}

\paragraph{Sensitivity Analysis} With the
$R = \mathbb{R} \cup \{\infty\}$ semiring, we let $\mathcal{W}$ be $R$
as well. The relation transformer is given by scaling. With a base
type of real numbers with relational intepretation
$R_{\mathrm{real}}~k~(x,x') \Leftrightarrow |x-x'| \leq k$, then the
fundamental lemma states that the usage annotations on the input
variables tracks the extent to which the program is sensitive to
changes in those variables.

\vspace{-0.6em}

\paragraph{Information Flow} With the $R = \mathcal{P}(L)$ semiring,
we again take $\mathcal{W} = R$, and let the relation transformer to
be
$\oc_l~R~l' = \{\top~\textrm{when }l \geq l'; R~\textrm{otherwise}\}$.
Then the fundamental lemma yields the same non-interference properties
as stated by Abadi et al. for the DCC \cite{abadi99core}.


% Local Variables:
% TeX-master: "quantitative"
% End:


\newpage

\appendix


\section{Introduction}

The Story:
\begin{enumerate}
\item We want to have a way of describing DSLs that are purely
  functional programs at heart, but have some special intensional
  properties stemming from restrictions on how they use the resources
  they are handed. Examples:
  \begin{enumerate}
  \item Linear type systems that ensure preservation of data
  \item Monotonicity
  \item Information flow
  \item Metric spaces and input sensitivity
  \end{enumerate}
\item We track how resources are used by using a ``quantitative
  coeffect'' system.
\item We build our system in stages, enabling reflective proof
  procedures that lift us from well-scoped terms to well-typed terms
  to well-resourced terms.
\item Well typed terms are equipped with a standard semantics: types
  are interpreted as (Agda) Sets, and terms are interpreted as Agda
  functions.
\item The resource typing permits more refined semantics, and this is
  where the fun starts. We give a general semantics that can be
  instantiated to any of our examples. These semantics capture (a) the
  constrained resource behaviour of programs (via Kripke indexing);
  and (b) the constrained observational behaviour of programs (via
  relations); both of these are consequences of the types.
\item Give some examples of the theorems that result from our
  semantics.
\item (everything has been formalised in Agda)
\end{enumerate}

\section{Background}

\begin{enumerate}
\item Petricek and Orchard
\item Brunel et al
\item Ghica and Smith
\item Benton/Kennedy/Hofmann relational refinements
\item Atkey/Johann/Kennedy algebraic indexed types
\item Gaboardi et al
\item Reed and Pierce
\end{enumerate}

\section{Well-scoped and Well-typed Programs}

\subsection{Well-scoped Syntax}

We present our theory as a system of three layers. First, we have well scoped
syntax. On top of this we impose typing discipline, and on top of that,
resourcing discipline.

The well scoped syntax is formalised using de Bruijn indices, but written here
with names for clarity. Justified by the de Bruijn formalisation, we assume that
all variable names are distinct.

The theory contains linear functions, bangs, tensor and with products, sums, and
units for the latter three. Intuitively, $\excl{\rho}{S}$ can be interpreted as
$\rho$-many copies of $S$, $\tensor{S}{T}$ as a pair containing both an $S$ and a
$T$, $\withT{S}{T}$ as either an $S$ or a $T$ (consumer's choice), and
$\sumT{S}{T}$ as either an $S$ or a $T$ (producer's choice). $\tensor{S}{T}$ is
a pair eliminated by pattern matching, whereas $\withT{S}{T}$ is a pair
eliminated by projections.

\begin{displaymath}
  \begin{array}{rrll}
    \pi,\rho & \in & R & \textrm{resource annotations} \\
    i & \in & \{0,1\} & \textrm{sides} \\
    \\
    S,T &  ::= & \fun{S}{T}                    & \textrm{function} \\
        & \mid & \excl{\rho}{S}                & \textrm{bang} \\
        & \mid & \tensorOne \mid \tensor{S}{T} & \textrm{tensor product} \\
        & \mid & \withTOne \mid \withT{S}{T}   & \textrm{with} \\
        & \mid & \sumTZero \mid \sumT{S}{T}    & \textrm{sum} \\
    \\
    s &  ::= & \lam{x}{s} & \textrm{lambda abstraction} \\
      & \mid & \bang{s} & \textrm{bang introduction} \\
      & \mid & \unit \mid \ten{s_0}{s_1} & \textrm{tensor product} \\
      &      &                           & \textrm{introduction} \\
      & \mid & \eat \mid \wth{s_0}{s_1} & \textrm{with introduction} \\
      & \mid & \inj{i}{s} & \textrm{sum introduction} \\
    e &  ::= & x & \textrm{variable} \\
      & \mid & \bm{T}{e}{\bind{x}{s}}  & \textrm{pattern matching} \\
      &      &                         & \textrm{for bang} \\
      & \mid & \del{T}{e}{s} & \textrm{deletion of tensor unit} \\
      & \mid & \prm{T}{e}{\bind{x,y}{s}} & \textrm{pattern matching} \\
      &      &                           & \textrm{for tensor products} \\
      & \mid & \proj{i}{e} & \textrm{projection for with} \\
      & \mid & \exf{T}{e} & \textrm{absurd elimination} \\
      & \mid & \cse{T}{e}{\bind{x}{s_0}}{\bind{y}{s_1}}
                       & \textrm{case analysis for sums}
  \end{array}
\end{displaymath}

\subsection{Well-typed}

Typing and resourcing rules are presented in \autoref{fig:typing-rules}. Typing
rules are given by ignoring all of the red annotations (superscripts and
hypotheses about $\Delta$).

Types enable us to give a total functional semantics to terms. This semantics
(presented in \autoref{fig:set-semantics}) does not distinguish between the two
notions of product, and treats $\excl{\rho}{S}$ the same as just $S$.

FIXME:
Is it possible to make Quantitative.Types have its own notion of types
that we can show is related to the resourced types? (the need for
special constructs for introducing and eliminating bangs is a wrinkle
here).

\begin{figure}
  \begin{mathpar}
    % Variables
    \inferrule{(x : S) \in \Gamma
               \\ \rescomment{\Delta \leq 0, x^1, 0}}
              {\ctx{\Gamma}{\Delta} \vdash x \in S}

    % Embedding
    \inferrule{\ctx{\Gamma}{\Delta} \vdash S \ni s}
              {\ctx{\Gamma}{\Delta} \vdash (\ann{s}{S}) \in S}
              
    \inferrule{\ctx{\Gamma}{\Delta} \vdash e \in S}
              {\ctx{\Gamma}{\Delta} \vdash S \ni \emb{e}}

    % Functions
    \inferrule{\ctx{\Gamma}{\Delta}, \ctxvar{x}{S}{1} \vdash T \ni s[x]}
              {\ctx{\Gamma}{\Delta} \vdash \fun{S}{T} \ni \lam{x}{s[x]}}

    \inferrule{\ctx{\Gamma}{\Delta_e} \vdash e \in \fun{S}{T}
               \\ \ctx{\Gamma}{\Delta_s} \vdash S \ni s
               \\ \rescomment{\Delta \leq \Delta_e + \Delta_s}}
              {\ctx{\Gamma}{\Delta} \vdash \app{e}{s} \in T}

    % Bang
    \inferrule{\ctx{\Gamma}{\Delta_s} \vdash S \ni s
               \\ \rescomment{\Delta \leq \rho \cdot \Delta_s}}
              {\ctx{\Gamma}{\Delta} \vdash \excl{\rho}{S} \ni \bang{s}}

    \inferrule{\ctx{\Gamma}{\Delta_e} \vdash e \in \excl{\rho}{S}
               \\ \ctx{\Gamma}{\Delta_s}, \ctxvar{x}{S}{\rho} \vdash T \ni s[x]
               \\ \rescomment{\Delta \leq \Delta_e + \Delta_s}}
              {\ctx{\Gamma}{\Delta} \vdash \bm{T}{e}{\bind{x}{s[x]}} \in T}

    % Tensor unit
    \inferrule{\rescomment{\Delta \leq 0}}
              {\ctx{\Gamma}{\Delta} \vdash \tensorOne \ni \unit}

    \inferrule{\ctx{\Gamma}{\Delta_e} \vdash e \in \tensorOne 
               \\ \ctx{\Gamma}{\Delta_s} \vdash T \ni s
               \\ \rescomment{\Delta \leq \Delta_e + \Delta_s}}
              {\ctx{\Gamma}{\Delta} \vdash \del{T}{e}{s} \in T}

    % Tensor
    \inferrule{\ctx{\Gamma}{\Delta_0} \vdash S_0 \ni s_0
               \\ \ctx{\Gamma}{\Delta_1} \vdash S_1 \ni s_1
               \\ \rescomment{\Delta \leq \Delta_0 + \Delta_1}}
              {\ctx{\Gamma}{\Delta} \vdash \tensor{S_0}{S_1} \ni \ten{s_0}{s_1}}

    \inferrule{\ctx{\Gamma}{\Delta_e} \vdash e \in \tensor{S_0}{S_1}
               \\ \ctx{\Gamma}{\Delta_s}, \ctxvar{x}{S_0}{1}, \ctxvar{y}{S_1}{1} \vdash T \ni s[x,y]
               \\ \rescomment{\Delta \leq \Delta_e + \Delta_s}}
              {\ctx{\Gamma}{\Delta} \vdash \prm{T}{e}{\bind{x,y}{s[x,y]}} \in T}

    % With unit
    \inferrule{ }
              {\ctx{\Gamma}{\Delta} \vdash \withTOne \ni \eat}

    %(no \withTOne elim)

    % With
    \inferrule{\ctx{\Gamma}{\Delta} \vdash S_0 \ni s_0
               \\ \ctx{\Gamma}{\Delta} \vdash S_1 \ni s_1}
              {\ctx{\Gamma}{\Delta} \vdash \withT{S_0}{S_1} \ni \wth{s_0}{s_1}}

    \inferrule{\ctx{\Gamma}{\Delta} \vdash e \in \withT{S_0}{S_1}}
              {\ctx{\Gamma}{\Delta} \vdash \proj{i}{e} \in S_i}

    % Sum unit
    %(no \sumTZero intro)

    \inferrule{\ctx{\Gamma}{\Delta_e} \vdash e \in \sumTZero
               \\ \rescomment{\Delta \leq \Delta_e + \Delta_s}}
              {\ctx{\Gamma}{\Delta} \vdash \exf{T}{e} \in T}

    % Sum
    \inferrule{\ctx{\Gamma}{\Delta} \vdash S_i \ni s}
              {\ctx{\Gamma}{\Delta} \vdash \sumT{S_0}{S_1} \ni \inj{i}{s}}

    \inferrule{\ctx{\Gamma}{\Delta_e} \vdash e \in \sumT{S_0}{S_1}
               \\ \ctx{\Gamma}{\Delta_s}, \ctxvar{x}{S_0}{1} \vdash T \ni s_0[x]
               \\ \ctx{\Gamma}{\Delta_s}, \ctxvar{y}{S_1}{1} \vdash T \ni s_1[y]
               \\ \rescomment{\Delta \leq \Delta_e + \Delta_s}}
              {\ctx{\Gamma}{\Delta} \vdash \cse{T}{e}{\bind{x}{s_0[x]}}{\bind{y}{s_1[y]}} \in T}
  \end{mathpar}                                                                                  
  \caption{Typing and resourcing rules}
  \label{fig:typing-rules}
\end{figure}

\begin{figure}
  \begin{displaymath}
    \llbracket S \rrbracket = \ldots
  \end{displaymath}
  \caption{Semantics of typed terms in Set}
  \label{fig:set-semantics}
\end{figure}

\section{Resourced Terms}

Present the refinement of typed terms via resources.

\subsection{Partially-ordered Semirings}

\begin{definition}
  A \emph{partially ordered semiring} is a ...
\end{definition}

\subsection{Resourced Terms}

\section{Relational Semantics}

Plan: for terms that are well-typed and well-resourced, 

\section{Example Instantiations}

\subsection{Permutations}

\subsection{Information Flow}

Idea: we make a semiring from the bilattice of confidentiality and
integrity.

\subsection{Monotonicity}

\subsection{Metric Spaces and Sensitivity}

\section{Conclusions}

TODO:
\begin{enumerate}
\item Mixing with effects
\item Dependent types
\item Non-termination (could use a step indexed/topos of trees model)
\item Coeffect polymorphism (could we just do this? Needs a more
  complex model of types)
\item The layers of well-formed-ness be presented via a system of
  ornamentation?
\end{enumerate}

%% Acknowledgments
\begin{acks}                            %% acks environment is optional
                                        %% contents suppressed with 'anonymous'
  %% Commands \grantsponsor{<sponsorID>}{<name>}{<url>} and
  %% \grantnum[<url>]{<sponsorID>}{<number>} should be used to
  %% acknowledge financial support and will be used by metadata
  %% extraction tools.
  % This material is based upon work supported by the
  % \grantsponsor{GS100000001}{National Science
  %   Foundation}{http://dx.doi.org/10.13039/100000001} under Grant
  % No.~\grantnum{GS100000001}{nnnnnnn} and Grant
  % No.~\grantnum{GS100000001}{mmmmmmm}.  Any opinions, findings, and
  % conclusions or recommendations expressed in this material are those
  % of the author and do not necessarily reflect the views of the
  % National Science Foundation.
  James Wood is supported by a EPSRC award (FIXME).
\end{acks}


%% Bibliography
\bibliography{bib}


%% Appendix
% \appendix
% \section{Appendix}

% Text of appendix \ldots

\end{document}
