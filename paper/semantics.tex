\paragraph{Underlying Semantics} We give a standard semantics of types
and well typed terms into sets and functions. This semantics ignores
the usage information. For types, we have:
\begin{displaymath}
  \begin{array}{ll}
    \llbracket \base \rrbracket = A_\base \\
    \llbracket \fun{S}{T} \rrbracket = \llbracket S \rrbracket \rightarrow \llbracket T \rrbracket &
    \llbracket \excl{\rho}{S} \rrbracket = \llbracket S \rrbracket \\
    \llbracket \tensorOne \rrbracket = \llbracket \withTOne \rrbracket = \{*\} &
    \llbracket \tensor{S}{T} \rrbracket = \llbracket \withT{S}{T} \rrbracket = \llbracket S \rrbracket \times \llbracket T \rrbracket \\
    \llbracket \sumTZero \rrbracket = \{\} &
    \llbracket \sumT{S}{T} \rrbracket = \llbracket S \rrbracket \uplus \llbracket T \rrbracket \\
  \end{array}
\end{displaymath}
Contexts are interpreted as left-nested products. Terms are assigned
the usual semantics as functions $\sem{t} : \sem{\Gamma} \to \sem{S}$.

\paragraph{Usage-tracking semantics} To derive interesting properties
from our type system, we refine the set-theoretic semantics by
Kripke-indexed binary relations. This gives a fundamental lemma for
our system, that when instantitated in different ways captures the
examples in the introduction.

Our framework is parameterised by a category $\mathcal{W}$ of
\emph{possible worlds} that track how resources are distributed by
programs. To interpret resource separation, we assume that
$\mathcal{W}$ has \emph{symmetric promonoidal structure}: profunctors
$J : 1 \nrightarrow \mathcal{W}$ and
$P : \mathcal{W} \times \mathcal{W} \nrightarrow \mathcal{W}$ such
that $P \odot (J \times 1) \cong 1$, $P \odot (1 \times J) \cong 1$,
$P \odot (1 \times P) \cong P \odot (P \times 1)$, and
$P \cong P \odot (\pi_2 \times \pi_1)$, and the triangle, pentagon,
and hexagon laws hold\footnote{We don't need the laws to hold to prove
  the fundamental lemma.}.

We now assign to each type $T$ a functor
$\sem{T}^R : \mathcal{W}^{op} \to \mathrm{Rel}~\sem{T}$ that captures
a notion of $\mathcal{W}$-indexed ``indistinguishability''. To
interpret $\oc_\rho S$, we assume we are given a relation transformer
$\oc_A : R^{op} \to \mathrm{Rel}(A)^{\mathcal{W}^{op}} \to
\mathrm{Rel}(A)^{\mathcal{W}^{op}}$
that satisfies the axioms of a monoidal exponential comonad. The
interesting cases are for functions, $\otimes$-products and the
$\oc_\rho$ modality:
\begin{displaymath}
  \begin{array}{l}
    \llbracket \fun{S}{T} \rrbracket^R~w~(f,f') = \\
    \quad \forall x,y.~P(y,w)x \Rightarrow \forall s,s'.~\llbracket S \rrbracket^R y (s, s') \Rightarrow \llbracket
    T \rrbracket^R x (f~s, f'~s')
    \\
    \llbracket \tensor{S}{T} \rrbracket^R~w~((s, t), (s', t')) =\\
    \quad
    \exists x,y.~P(x,y)w \wedge \llbracket S \rrbracket^R x (s, s') \wedge
    \llbracket T \rrbracket^R y (t, t')
    \\
    \llbracket \excl{\rho}{S} \rrbracket^R~w~(s,s')=
    \oc_\rho \llbracket S \rrbracket^R~w~(s,s')\\
  \end{array}
\end{displaymath}
Contexts
$\ctxvar{x_1}{S_1}{\rho_1}, \ldots, \ctxvar{x_n}{S_n}{\rho_n}$ are
interpreted as if they were
$\sem{(\cdots(1 \otimes \oc_{\rho_1}S_1) \cdots \otimes \oc_{\rho_n}
  S_n)}$.
With these definitions, we can prove the following fundamental lemma
for our Kripke-indexed relational semantics.

% $R \times S$ and $R \uplus S$ are defined pointwise on relations.
% Particularly, there are two cases for $R \uplus S$:

% \begin{itemize}
%   \item $R(r, r')$ implies $(R \uplus S)(\mathrm{inl}~r, \mathrm{inl}~r')$.
%   \item $S(s, s')$ implies $(R \uplus S)(\mathrm{inr}~s, \mathrm{inr}~s')$.
% \end{itemize}

% We assume a family of natural transformations $\oc$ satisfying the following laws.

%   \begin{mathpar}
%     \rho \leq \pi \implies (\oc_\pi R~w \implies \oc_\rho R~w) \and
%     \oc_0 R~w \implies J w \and
%     \oc_{\rho+\pi} R~w~(a, b) \implies \exists x,y. P(x,y)w \wedge \oc_\rho R~x~a \wedge \oc_\pi R~y~b \and
%     \oc_1 R~w \iff R~w \and
%     \oc_{\rho \cdot \pi} R~w \iff \oc_\rho(\oc_\pi R)~w
%   \end{mathpar}

% The semantics of a context $\ctxvar{x_1}{S_1}{\rho_1}, \ldots, \ctxvar{x_n}{S_n}{\rho_n}$ is given by $\tensor{\llbracket \excl{\rho_1}{S_1}}{\tensor{\ldots}{\excl{\rho_n}{S_n}}} \rrbracket^R$.

% This indexed relational semantics gives us a family of logical relations.
% The fundamental lemma is as follows.

\begin{theorem}[Fundamental Lemma]
  \begin{displaymath}
    \ctx{\Gamma}{\Delta} \vdash t : T \implies \llbracket \ctx{\Gamma}{\Delta} \rrbracket^R w~(\gamma, \gamma') \implies \llbracket T \rrbracket^R w~(\sem{t}\gamma, \sem{t}\gamma')
  \end{displaymath}
\end{theorem}


% Local Variables:
% TeX-master: "quantitative"
% End:
