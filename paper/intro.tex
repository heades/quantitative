In normal typed $\lambda$-calculi, variables may be used multiple
times, in multiple contexts, for multiple reasons, as long as the
types agree. The disciplines of linear types \cite{girard87linear} and
coeffects \cite{PetricekOM14,BrunelGMZ14,GhicaS14} refine this by
keeping track of how variables are used throughout a program. For
instance, we might track how many times a variable is used, or whether
it is used covariantly, contravariantly, or invariantly. Coeffects
give us a general framework of ``context constrained computing'',
where constraints on variables in the context tell us something
interesting about the computation being performed. Thus we put the
type information to work to tell us facts about programs that might
not otherwise be apparent.

We will present work in progress on capturing the ``intensional''
properties of programs via a family of Kripke indexed relational
semantics that refines a simple Set-theoretic semantics of
programs. The value of our approach lies in its generality. We can
accommodate the following examples:
\begin{enumerate}
\item Linear types that capture properties like ``all list
  manipulating programs are permutations''. This example uses the
  Kripke-indexing of relations to track the collection of datums
  currently being manipulated by the program.
\item Monotonicity coeffects that track whether a program uses its
  input co-, contra-, or in-variantly (or not at all). In this case,
  the Kripke-indexing is not used, but the coeffect annotation
  describes how to manipulate the relations.
\item Sensitivity typing, that tracks how sensitive the output of the
  program is in terms of changes to the input. This forms the core of
  type systems for differential privacy \cite{reed10distance}.
\item Information flow typing, in the style of the Dependency Core
  Calculus \cite{abadi99core}.
\end{enumerate}
Through discusssion at the workshop, we hope to discover more
applications of our framework. In future work, we plan to extend our
framework to encompass type dependency, and to explore the space of
possible inductive data types and elimination principles possible in
the presence of usage information.

The syntax and semantics we present here have been formalised in Agda:
\url{https://github.com/laMudri/quantitative/}. Formalisation of the
examples is in progress.

% Local Variables:
% TeX-master: "quantitative"
% End:
