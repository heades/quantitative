We define our type system with respect to a partially ordered semiring
$R$ for tracking how variables are used. A \emph{partially ordered
  semiring} $(R, \leq, 0, +, 1 , \cdot)$ is a poset $(R, \leq)$,
commutative monoid $(R, 0, +)$, and monoid $(R, 1, \cdot)$, such that
$\cdot$ distributes over $0$ and $+$, and $+$ and $\cdot$ are
monotonic with respect to $\leq$. We take $\rho,\pi \in R$.

\paragraph{Examples}
\begin{inparaenum}
\item The zero-one-many semiring $\{0,1,\omega\}$ simulates linear
  typing in our system.
\item Monotonicity typing uses the semiring with carrier
  $\{0,\uparrow,\downarrow,\equiv\}$, where the multiplicative unit is
  $\uparrow$ (covariance). The $\downarrow$ represents
  contra-variance, and $\equiv$ represents invariance.
\item Sensitivity analysis uses the semiring with carrier
  $\mathbb{R} \cup \{\infty\}$ with $\min$ and $+$ as the addition and
  multiplication.
\item Information flow analysis uses the semiring with carrier
  $\mathcal{P}(L)$, where $L$ is some set of labels.
\end{inparaenum}

\medskip

The base language we consider is a
bidirectional \cite{DBLP:journals/toplas/PierceT00} simply typed
$\lambda$ calculus with the following types, where $\iota$ ranges over
some set of base types:
\begin{displaymath}
  S,T ::= \fun{S}{T} \mid \excl{\rho}{S} \mid \tensorOne \mid \tensor{S}{T} \mid
  \withTOne \mid \withT{S}{T} \mid \sumTZero \mid \sumT{S}{T} \mid \iota
\end{displaymath}
Bidirectional typing type annotations required.  Since our language is
bidirectionally typed, we have two syntactic categories of terms: $e$
ranges over checked terms, and $s$ ranges over synthesising terms. We
use $t$ for both.
\begin{displaymath}
  \begin{aligned}
    s &  ::= \lam{x}{s} \mid \bang{s} \mid \unit \mid \ten{s_0}{s_1} \mid \eat \mid \wth{s_0}{s_1} \mid \inj{i}{s} \mid \emb{e} \\
    e &  ::= x \mid \app{e}{s} \mid \bm{T}{e}{\bind{x}{s}} \mid \del{T}{e}{s} \mid \prm{T}{e}{\bind{x,y}{s}} \\
      & \quad \mid \proj{i}{e} \mid \exf{T}{e} \mid
             \cse{T}{e}{\bind{x}{s_0}}{\bind{y}{s_1}} \mid \ann{s}{S}
  \end{aligned}
\end{displaymath}
where curly braces and $\lambda$ denote variable binding and we take
$i \in \{0,1\}$ wherever it appears.

Contexts $\Gamma$ assign to each variable a type $S$ and a usage
$\rho \in R$:
$\Gamma = \ctxvar{x_1}{S_1}{\rho_1}, \ldots,
\ctxvar{x_n}{S_n}{\rho_n}$.
Contexts whose variables and types match form a left $R$-semimodule,
by pointwise addition and scaling of the usage annotations. The
partial order on $R$ is extended pointwise to contexts.
% Contexts $\ctx{\Gamma}{\Delta}$ over variables $x_1, \ldots, x_n$ are to be understood as a \emph{typing context} $\Gamma$ of the form $x_1 : S_1, \ldots, x_n : S_n$, and a \emph{resourcing context} $\Delta$ of the form $x_0^{\rescomment{\rho_0}}, \ldots, x_n^{\rescomment{\rho_n}}$.
% %The resourcing context in which for each $k$, $\rho_k = 0$, will be abbreviated to $\rescomment{0}$.
% Resourcing contexts with matching variables form a left $R$-semimodule, taking $(x_0^{\rescomment{\rho_0}}, \ldots, x_n^{\rescomment{\rho_n}}) + (x_0^{\rescomment{\pi_0}}, \ldots, x_n^{\rescomment{\pi_n}}) = (x_0^{\rescomment{\rho_0 + \pi_0}}, \ldots, x_n^{\rescomment{\rho_n + \pi_n}})$ and $\rho \cdot (x_0^{\rescomment{\rho \cdot \pi_0}}, \ldots, x_n^{\rescomment{\rho \cdot \pi_n}})$.
Typing judgements for checked and synthesising terms have the same
contexts, but either record that a term is checked against a type
($\Gamma \vdash T \ni S$) or synthesise a type
($\Gamma \vdash e \in T$).
% When we only care about typability, we write $t : T$ in place of
% either $T \ni t$ or $t \in T$.

The typing rules consist of a variable rule, two rules for change of
direction, and introduction and elimination rules for each type
former. The following rules for variables, function introduction and
elimination, and $\oc_\rho$ introduction illustrate how usage
information is tracked:
\begin{mathpar}
  % Variables
  \inferrule
  {\Gamma \leq 0\Gamma_1, \ctxvar{x}{S}{1}, 0\Gamma_2}
  {\Gamma \vdash x \in S}

  % Functions
  \inferrule
  {\Gamma, \ctxvar{x}{S}{1} \vdash T \ni s[x]}
  {\Gamma \vdash \fun{S}{T} \ni \lam{x}{s[x]}}

  \inferrule
  {\Gamma_1 \vdash e \in \fun{S}{T}
    \\ \Gamma_2 \vdash S \ni s
    \\ \Gamma \leq \Gamma_1 + \Gamma_2}
  {\Gamma \vdash \app{e}{s} \in T}

  % Bang
  \inferrule
  {\Gamma_s \vdash S \ni s \\ \Gamma \leq \rho \cdot \Gamma_s}
  {\Gamma \vdash \excl{\rho}{S} \ni \bang{s}}

  %\inferrule{\ctx{\Gamma}{\Delta_e} \vdash e \in \excl{\rho}{S}
  %           \\ \ctx{\Gamma}{\Delta_s}, \ctxvar{x}{S}{\rho} \vdash T \ni s[x]
  %           \\ \rescomment{\Delta \leq \Delta_e + \Delta_s}}
  %          {\ctx{\Gamma}{\Delta} \vdash \bm{T}{e}{\bind{x}{s[x]}} \in T}
\end{mathpar}
Sub-resourcing, weakening (adding $0$-use variables to the context)
and substitution are all admissible.  In our Agda formalisation, we
have constructed our type system in two levels: a non-usage tracked
simply-typed $\lambda$-calculus, with a usage-tracking system layered
above. This emphasises the use of coeffect annotations as an
\emph{analysis} of programs, they do not affect the underlying
semantics, but comment on it. We introduce our semantic framework in
the next section.

%commented out for space

% The order acts as a sub-resourcing relation.
% Well resourced terms can also be weakened by the introduction of $0$-resource
% variables to the context, and satisfy a substitution principle.

% \begin{mathpar}
%   \inferrule*[Right=subres]
%              {\ctx{\Gamma}{\Delta} \vdash t : T
%                \\ \rescomment{\Delta' \leq \Delta}}
%              {\ctx{\Gamma}{\Delta'} \vdash t : T}

%   \inferrule*[Right=weak]
%              {\ctx{\Gamma}{\Delta}, \ctx{\Gamma'}{\Delta'} \vdash t : T}
%              {\ctx{\Gamma}{\Delta}, \ctxvar{x}{S}{0}, \ctx{\Gamma'}{\Delta'}
%                \vdash t : T}

%   \inferrule*[Right=subst]
%              {\ctx{\Gamma}{\Delta} \vdash t : T
%                \\ \forall x.~
%                \ctx{\Gamma'}{\vec{\Delta'_x}} \vdash \vec{t'_x} : \Gamma_x
%                \\ \rescomment{\Delta'' \leq \sum_x \Delta_x \cdot \vec{\Delta'_x}}}
%              {\ctx{\Gamma'}{\Delta''} \vdash t[\vec{t'}] : T}
% \end{mathpar}


% Local Variables:
% TeX-master: "quantitative"
% End:
