The language we consider is the bidirectional\cite{DBLP:journals/toplas/PierceT00} simply typed lambda
calculus\cite{Barendregt92lambdacalculi} with functions, indexed bangs, tensor products, with products,
sums, units for the products and sums, and a base type.

Our syntax is defined with respect to a partially ordered semiring $R$, and we take $\rho,\pi \in R$.

\begin{definition}
  A \emph{partially ordered semiring} $(R, \leq, 0, +, 1 , \cdot)$ is a poset $(R, \leq)$, commutative monoid $(R, 0, +)$, and monoid $(R, 1, \cdot)$, such that $\cdot$ distributes over $0$ and $+$, and $+$ and $\cdot$ are monotonic with respect to $\leq$.
\end{definition}

Additionally, we take $i \in \{0,1\}$ wherever it appears.
Variables are formalised as de Bruijn indices\cite{deBruijn:dummies}, but
presented named, with names $x$ and $y$.
Curly braces (and $\lambda$) denote the binding of new variables in the following expression.

\begin{displaymath}
  S,T ::= \fun{S}{T} \mid \excl{\rho}{S} \mid \tensorOne \mid \tensor{S}{T} \mid
  \withTOne \mid \withT{S}{T} \mid \sumTZero \mid \sumT{S}{T} \mid \iota
\end{displaymath}
\begin{displaymath}
  \begin{aligned}
    s &  ::= \lam{x}{s} \mid \bang{s} \mid \unit \mid \ten{s_0}{s_1} \\
      & \quad \mid \eat \mid \wth{s_0}{s_1} \mid \inj{i}{s} \mid \emb{e} \\
    e &  ::= x \mid \app{e}{s} \mid \bm{T}{e}{\bind{x}{s}} \mid \del{T}{e}{s} \mid \prm{T}{e}{\bind{x,y}{s}} \\
      & \quad \mid \proj{i}{e} \mid \exf{T}{e} \mid
             \cse{T}{e}{\bind{x}{s_0}}{\bind{y}{s_1}} \mid \ann{s}{S}
  \end{aligned}
\end{displaymath}

Contexts $\ctx{\Gamma}{\Delta}$ over variables $x_1, \ldots, x_n$ are to be understood as a \emph{typing context} $\Gamma$ of the form $x_1 : S_1, \ldots, x_n : S_n$, and a \emph{resourcing context} $\Delta$ of the form $x_0^{\rescomment{\rho_0}}, \ldots, x_n^{\rescomment{\rho_n}}$.
%The resourcing context in which for each $k$, $\rho_k = 0$, will be abbreviated to $\rescomment{0}$.
Resourcing contexts with matching variables form a left $R$-semimodule, taking $(x_0^{\rescomment{\rho_0}}, \ldots, x_n^{\rescomment{\rho_n}}) + (x_0^{\rescomment{\pi_0}}, \ldots, x_n^{\rescomment{\pi_n}}) = (x_0^{\rescomment{\rho_0 + \pi_0}}, \ldots, x_n^{\rescomment{\rho_n + \pi_n}})$ and $\rho \cdot (x_0^{\rescomment{\rho \cdot \pi_0}}, \ldots, x_n^{\rescomment{\rho \cdot \pi_n}})$.

Our typing and resourcing judgements will be in one of the following two forms.
\begin{displaymath}
  \begin{array}{l}
    \ctxvar{x_1}{S_1}{\rho_1}, \ldots, \ctxvar{x_n}{S_n}{\rho_n} \vdash T \ni s \\
    \ctxvar{x_1}{S_1}{\rho_1}, \ldots, \ctxvar{x_n}{S_n}{\rho_n} \vdash e \in T
  \end{array}
\end{displaymath}

$t$ ranges over $s$ and $e$, and we may write $t : T$ in place of either $T \ni
t$ or $t \in T$.

There is a variable rule, two rules for change of direction, and introduction and elimination rules for each type former: functions, bangs, and nullary and binary tensor products, with products, and sums.
For example, here are the rules for variables, function introduction and elimination, and bang introduction.

\begin{mathpar}
  % Variables
  \inferrule{(x : S) \in \Gamma
           \\ \rescomment{\Delta \leq 0, x^1, 0}}
          {\ctx{\Gamma}{\Delta} \vdash x \in S}

  % Functions
  \inferrule{\ctx{\Gamma}{\Delta}, \ctxvar{x}{S}{1} \vdash T \ni s[x]}
            {\ctx{\Gamma}{\Delta} \vdash \fun{S}{T} \ni \lam{x}{s[x]}}

  \inferrule{\ctx{\Gamma}{\Delta_e} \vdash e \in \fun{S}{T}
             \\ \ctx{\Gamma}{\Delta_s} \vdash S \ni s
             \\ \rescomment{\Delta \leq \Delta_e + \Delta_s}}
            {\ctx{\Gamma}{\Delta} \vdash \app{e}{s} \in T}

  % Bang
  \inferrule{\ctx{\Gamma}{\Delta_s} \vdash S \ni s
             \\ \rescomment{\Delta \leq \rho \cdot \Delta_s}}
            {\ctx{\Gamma}{\Delta} \vdash \excl{\rho}{S} \ni \bang{s}}

  %\inferrule{\ctx{\Gamma}{\Delta_e} \vdash e \in \excl{\rho}{S}
  %           \\ \ctx{\Gamma}{\Delta_s}, \ctxvar{x}{S}{\rho} \vdash T \ni s[x]
  %           \\ \rescomment{\Delta \leq \Delta_e + \Delta_s}}
  %          {\ctx{\Gamma}{\Delta} \vdash \bm{T}{e}{\bind{x}{s[x]}} \in T}
\end{mathpar}                                                                                  

% Local Variables:
% TeX-master: "quantitative"
% End:
