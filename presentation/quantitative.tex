\documentclass{beamer}

\usepackage{booktabs}
\usepackage{subcaption}

\usepackage{stmaryrd}
\usepackage{mathpartir}
\usepackage{amssymb}
\usepackage{cmll}
\usepackage{xcolor}
\usepackage{makecell}
\usepackage{tikz-cd}

\usetikzlibrary{positioning}
%\usetikzlibrary{decorations.markings}
\usetikzlibrary{arrows}

\def\newelims{1}
\ifx\newelims\undefined
  \def\newelims{0}
\fi

\newcommand{\bind}[2]{#2}
%\newcommand{\ctx}[2]{{#1}^{\rescomment{#2}}}
\newcommand{\ctx}[2]{{#2}^{#1}}
\newcommand{\ctxvar}[3]{#1 \stackrel{\rescomment{#3}}: #2}
%\newcommand{\rescomment}[1]{{\color{red}#1}}
\newcommand{\rescomment}[1]{{#1}}


\newcommand{\ann}[2]{#1 : #2}
\newcommand{\emb}[1]{\underline{#1}}

\newcommand{\base}[0]{\iota}

\newcommand{\fun}[2]{#1 \multimap #2}
\newcommand{\lam}[2]{\lambda #1.~#2}
\newcommand{\app}[2]{#1~#2}

\newcommand{\excl}[2]{\oc_{#1} #2}
\newcommand{\bang}[1]{\operatorname{bang} #1}
\newcommand{\bm}[4]{\if\newelims0%
\operatorname{bm}_{#1}(#2, \bind{#3}{#4})%
\else%
\mathrm{let}~\bang{#3} = #2~\mathrm{in}~#4 : #1%
\fi}

\newcommand{\tensorOne}[0]{1}
\newcommand{\unit}[0]{{*}_\otimes}
\newcommand{\del}[3]{\if\newelims0%
\operatorname{del}_{#1}(#2, #3)%
\else%
\mathrm{let}~\unit = #2~\mathrm{in}~#3 : #1%
\fi}

\newcommand{\tensor}[2]{#1 \otimes #2}
\newcommand{\ten}[2]{(#1, #2)_{\otimes}}
\newcommand{\prm}[5]{\if\newelims0%
\operatorname{pm}_{#1}(#2, \bind{#3, #4}{#5})%
\else%
\mathrm{let}~\ten{#3}{#4} = #2~\mathrm{in}~#5 : #1%
\fi}

\newcommand{\withTOne}[0]{\top}
\newcommand{\eat}[0]{{*}_{\with}}

\newcommand{\withT}[2]{#1 \with #2}
\newcommand{\wth}[2]{(#1, #2)_{\with}}
\newcommand{\proj}[2]{\operatorname{proj}_{#1} #2}

\newcommand{\sumTZero}[0]{0}
\newcommand{\exf}[2]{\operatorname{ex-falso}_{#1}(#2)}

\newcommand{\sumT}[2]{#1 \oplus #2}
\newcommand{\inj}[2]{\operatorname{inj}_{#1} #2}
\newcommand{\cse}[6]{\if\newelims0%
\operatorname{case}_{#1}(#2, \bind{#3}{#4}, \bind{#5}{#6})%
\else%
\mathrm{case}~#2~\mathrm{of}~\inj{0}{#3} \mapsto #4; \inj{1}{#5} \mapsto #6%
\fi}


\newcommand{\typed}[1]{\mathit{#1t}}
\newcommand{\resourced}[1]{\mathit{#1r}}


\newcommand{\sem}[1]{\llbracket #1 \rrbracket}

%\def\tobar{\mathrel{\mkern3mu  \vcenter{\hbox{$\scriptscriptstyle+$}}%
%                    \mkern-12mu{\to}}}
\newcommand\tobar{\mathrel{\ooalign{\hfil$\mapstochar\mkern5mu$\hfil\cr$\to$\cr}}}


\title{Context Constrained Computation}
\subtitle{via a linear-like lambda calculus}
\author{Bob Atkey\inst{1} \and James Wood\inst{1}}
\institute{\inst{1}University of Strathclyde}
\date{TyDe Workshop, 2018}

\begin{document}
  \frame{\titlepage}
  \begin{frame}
    \frametitle{Motivation}

    \begin{itemize}
    \item Constrain how variables are used
    \item Derive more free theorems about constrained programs
    \item Generalise the ``how many'' of linear typing \pause
      \begin{itemize}
      \item At what security level? -- information flow \pause
      \item How far away? -- sensitivity analysis \pause
      \item In which direction? -- monotonicity \pause
      \end{itemize}
    \item Formalised in Agda -- the free theorems of the object language are
      available to Agda programs
    \end{itemize}
  \end{frame}
  \begin{frame}
    \frametitle{Fundamental lemma}
    \begin{itemize}
    \item $\llbracket t \rrbracket : \llbracket \Gamma \rrbracket \to
      \llbracket S \rrbracket$
    \item $\forall w.~\forall(\gamma, \gamma') \in \llbracket \Gamma
      \rrbracket^R~w.~(\llbracket t \rrbracket~\gamma, \llbracket t
      \rrbracket~\gamma') \in \llbracket S \rrbracket^R~w$
    \item Consequences:
      \begin{itemize}
      \item Worlds are bags of keys, semiring counts usages \\
        $\implies$ all functions are permutations
      \item Worlds are distances, semiring tracks distances \\
        $\implies$ all functions are non-expansive
      \item Worlds are trivial, semiring tracks polarity \\
        $\implies$ all functions are monotonic
      \end{itemize}
    \end{itemize}
  \end{frame}
  \begin{frame}
    \frametitle{Metasyntax}
    \begin{itemize}
    \item Partially ordered semiring: $(R, \leq, 0, +, 1, \cdot)$, general elements $\rho, \pi$ \pause
    \item Contexts
      \begin{itemize}
      \item Scopes $m$, $n$ (natural numbers)
      \item typing contexts $\Gamma = x : S, \ldots, y : T$
      \item resourcing contexts $\Delta = x^\pi, \ldots, y^\rho$ \pause
      \end{itemize}
    \item Bidirectional two-level typing
      \begin{itemize}
      \item Well scoped synthesising terms $e$ and checkable terms $s$
      \item $t$ ranges over $e$ and $s$. \pause
      \item Synthesis: $\typed e : (\Gamma \vdash e \in S)$ \pause
      \item Checking: $\typed s : (\Gamma \vdash S \ni s)$ \pause
      \item Either: $\typed t : (\Gamma \vdash t : S)$ \pause
      \item Resourcing: $\resourced t : (\Delta \vdash \typed t)$ \pause
      \item Abbreviations $\ctx{\Gamma}{\Delta} \vdash e \in S$,
        $\ctx{\Gamma}{\Delta} \vdash S \ni s$, etc. \pause
      \end{itemize}
    %\item Simultaneous substitution: $t[\sigma]$, $\typed t[\typed \sigma]$,
    %  $\resourced t[\resourced \sigma]$
    \end{itemize}
  \end{frame}
  \begin{frame}
    \frametitle{With product}
    \begin{itemize}
    \item Introduction:
      \inferrule{\ctx{\Gamma}{\Delta} \vdash S_0 \ni s_0
                 \\ \ctx{\Gamma}{\Delta} \vdash S_1 \ni s_1}
                {\ctx{\Gamma}{\Delta} \vdash \withT{S_0}{S_1} \ni \wth{s_0}{s_1}}
      \bigskip
    \item Elimination:
      \inferrule{\ctx{\Gamma}{\Delta} \vdash e \in \withT{S_0}{S_1}
                 \\ i \in \{0,1\}}
                {\ctx{\Gamma}{\Delta} \vdash \proj{i}{e} \in S_i}
      \bigskip
    \item Negative type
    \end{itemize}
  \end{frame}
  \begin{frame}
    \frametitle{Tensor product}
    \begin{itemize}
    \item Introduction:
      \inferrule{\ctx{\Gamma}{\Delta_0} \vdash S_0 \ni s_0
                 \\ \ctx{\Gamma}{\Delta_1} \vdash S_1 \ni s_1
                 \\\\ \rescomment{\Delta \leq \Delta_0 + \Delta_1}}
                {\ctx{\Gamma}{\Delta} \vdash \tensor{S_0}{S_1} \ni \ten{s_0}{s_1}}
      \bigskip
    \item Elimination:
      \inferrule{\ctx{\Gamma}{\Delta_e} \vdash e \in \tensor{S_0}{S_1}
                 \\\\ \ctx{\Gamma}{\Delta_s},
                      \ctxvar{x}{S_0}{1}, \ctxvar{y}{S_1}{1} \vdash \bind{x,y}{s} \in T
                 \\\\ \rescomment{\Delta \leq \Delta_e + \Delta_s}}
                {\ctx{\Gamma}{\Delta} \vdash \prm{T}{e}{x}{y}{s} \in T}
      \bigskip
    \item Positive type
    \end{itemize}
  \end{frame}
  \begin{frame}
    \frametitle{Bang}
    \begin{itemize}
    \item Introduction:
      \inferrule{\ctx{\Gamma}{\Delta_s} \vdash S \ni s
                 \\ \rescomment{\Delta \leq \rho \cdot \Delta_s}}
                {\ctx{\Gamma}{\Delta} \vdash \excl{\rho}{S} \ni \bang{s}}
    \item Elimination:
      \inferrule{\ctx{\Gamma}{\Delta_e} \vdash e \in \excl{\rho}{S}
                 \\ \ctx{\Gamma}{\Delta_s}, \ctxvar{x}{S}{\rho}
                    \vdash T \ni \bind{x}{s}
                 \\ \rescomment{\Delta \leq \Delta_e + \Delta_s}}
                {\ctx{\Gamma}{\Delta} \vdash \bm{T}{e}{x}{s} \in T}
    \item Graded comonad:
      \begin{mathpar}
        \excl{1}{A} \to A
        \\
        \excl{\pi \cdot \rho}{A} \to \excl{\pi}{\excl{\rho}{A}}
      \end{mathpar}
    \end{itemize}
  \end{frame}
  \begin{frame}
    \frametitle{Substitution}
    \begin{itemize}
    \item<1-> Well scoped substitution
      \begin{flalign*}
        m \Rightarrow n :\equiv (x \in n) \to \mathrm{Tm}~m~\mathrm{syn}
      \end{flalign*}
    \item<2-> Typed substitution refines scoped substitution
      ($\sigma : m \Rightarrow n$)
      \begin{flalign*}
        \Gamma_m \Rightarrow_\sigma^t \Gamma_n :\equiv
        \left((x : T) \in \Gamma_n\right) \to \Gamma_m \vdash \sigma~x \in T
      \end{flalign*}
    \item<3-> Resourced substitution refines typed substitution
      ($\sigma t : \Gamma_m \Rightarrow_\sigma^t \Gamma_n$)
      \begin{flalign*}
        \Delta_m \Rightarrow_{\typed \sigma}^r \Delta_n :\equiv {}
        \only<3>{
          &(\Delta' : n \to \mathop{RCtx} m) \\
          &\times \left(\Delta_m \leq \sum_{x^\rho \in \Delta_n}\rho \cdot \Delta'_x\right)
        } \only<4>{
          &(\Delta' : \mathrm{Mat}~R~(m, n) \\
          &\times \left(\Delta_m \leq \Delta'\Delta_n\right)
        } \\
        &\times \left((x^\rho \in \Delta_n) \to \Delta'_x \vdash \typed \sigma~x\right)
      \end{flalign*}
    \end{itemize}
  \end{frame}
  \begin{frame}
    \frametitle{Resourced substitution example}
    \begin{itemize}
    \item
      \begin{flalign*}
        \Delta_m \Rightarrow_{\typed \sigma}^r \Delta_n :\equiv {}
        &(\Delta' : \mathrm{Mat}~R~(m, n) \\
        &\times \left(\Delta_m \leq \Delta'\Delta_n\right) \\
        &\times \left((x^\rho \in \Delta_n) \to \Delta'_x \vdash \typed \sigma~x\right)
      \end{flalign*}
    \item Variable rule
      \inferrule{\ctx{\Gamma}{\Delta} \leq \underline 0, \ctxvar{x}{S}{1}, \underline 0}
                {\ctx{\Gamma}{\Delta} \vdash x \in S}
    \item<2->
      $\mathrm{id}_\Delta :\equiv (\Delta', \mathit{prf}, \mathrm{var}) \quad
      \textrm{where}~\Delta'_x :\equiv \underline 0, x^1, \underline 0$
    \item<3-> $\mathit{prf}$ is $\Delta = I\Delta = \Delta'\Delta$
    \end{itemize}
  \end{frame}
  \begin{frame}
    \frametitle{Fundamental lemma, revisited}
    \begin{itemize}
    \item $\llbracket tt \rrbracket : \llbracket \Gamma \rrbracket \to
      \llbracket S \rrbracket$
    \item $\llbracket \ctx{\Gamma}{\Delta} \rrbracket^R : \mathcal{W} \to
      \llbracket \Gamma \rrbracket \times \llbracket \Gamma \rrbracket \to \Omega$
    \item $\forall w.~\forall(\gamma, \gamma') \in \llbracket \ctx{\Gamma}{\Delta}
      \rrbracket^R~w.~(\llbracket tt \rrbracket~\gamma, \llbracket tt
      \rrbracket~\gamma') \in \llbracket S \rrbracket^R~w$
    \item Consequences:
      \begin{itemize}
      \item Worlds are bags of keys, semiring counts usages \\
        $\implies$ the keys appearing in the context are a permutation of the
        keys appearing in the term
      \item Worlds specify maximum distance, semiring tracks distances \\
        $\implies$ the term is sensitive to changes in variables proportional to
        the annotations on those variables
      \item Worlds are trivial, semiring tracks polarity \\
        $\implies$ if assignments in the environment increase/decrease, the
        result follows suit
      \end{itemize}
    \end{itemize}
  \end{frame}
  \begin{frame}
    \frametitle{Conclusion}
    \begin{itemize}
    \item Reed, Pierce 2010 -- Distance Makes the Types Grow Stronger
      % More references
    \item \url{https://github.com/laMudri/quantitative}
    \end{itemize}
  \end{frame}
\end{document}
